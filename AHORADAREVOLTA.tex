\part{A Hora da Revolta}

\chapter*{}

\epigraph{``Pessimismo da ação, otimismo da vontade''}{Antonio Gramsci}


\newcommand{\falaG}{\itshape}
\newcommand{\falaM}{\normalfont}

\parindent0pt
\parskip4pt

\versal{Gianni} -- \itshape Vivemos sob um Estado de exceção em que golpes nascem dentro de
golpes. Luiz Inacio Lula da Silva, ou o candidato/a indicado por ele
caso seja condenado em segunda instância, vencerá o pleito Presidencial
de outubro de 2018. Visto que este opúsculo leva o título \emph{A Hora
da Revolta}, como arrebanhar o povo?

\normalfont\versal{Mino} -- \normalfont Antes de o Lula ser condenado em segunda instância, é preciso
organizar uma operação de largo espectro, capilar, chegando em todos os
cantos do País. O Lula tem perfeitamente condições de fazer isso. E o
Partido dos Trabalhadores tem de deixar de ser um partido inerte. E
preciso organizar uma ampla campanha, inclusive nas redes sociais, mas
principalmente nas ruas. E todas as figuras progressistas e forças
democráticas têm de forjar sólidos elos com o PT. Bastaria isso. Já
lancei a ``Operação Misericórdia pelo Brasil'' me dirigindo aos leitores
de \emph{CartaCapital}. Os leitores deveriam ter um impacto sobre todos
aqueles que puderem alcançar. Por favor, expliquem a eles o que está
acontecendo no neste País. Não é possível levar a vida como se a
situação que estamos atravessando fosse uma situação normal. Estamos
vivendo golpes dentro de golpes. É hora de reagir.

\falaG O golpe institucional para destituir Dilma remonta a agosto de 2016.
O objetivo era tornar o Lula Inelegível.

\falaM No meu entendimento é preciso compreender que retirar o Lula desse
páreo sem provas por meio de uma condenação culminada pelo Tribunal do
Santo Ofício de Curitiba implica um pleito ilegítimo. Esse é primeiro
ponto a ser encarado. Sem Lula esse pleito não vale nada. Esse é o ponto
central a meu ver. Agora é claro que a situação implica riscos graves.
Por quê? Porque se você observar o futuro eleitoral percebe que nem o
PMDB nem o tucanato têm condições de eleger um candidato capaz de
garantir a sobrevida da casa"-grande e da senzala. Por isso haveria um
golpe dentro do golpe para impedir as eleições de 2018. E com o apoio
dos Estados Unidos da América.

\falaG Lula provavelmente será condenado em segunda instância. O magistrado
rio"-grandense é padrinho de casamento de Moro e fez mestrado com o
mesmo. Além disso, como sabemos, o Sul é terra de maioria conservadora e
não podemos dizer que naquela região prima um grande apreço pelo Lula.

\falaM Não há saída para o País sem um enfrentamento. Um grande abalo
social. Forte. Até um terremoto, se for o caso. Seria mais que um forte
abalo forte. Lula já acreditou na possibilidade da conciliação, mas
acredito que tenha finalmente entendido que conciliação é impossível. Os
Pactos de la Moncloa que houve na Espanha após a morte de Franco em 1975
estabilizaram o processo de transição para uma democracia. Foram
conduzidos por um homem de direita que, no entanto, admiro bastante,
Adolfo Suárez. Pois bem, os Pactos de la Moncloa, de outubro de 1977, no
Brasil são totalmente impossíveis porque a casa"-grande não aceita
conversar com o trabalhador. Veja como a mídia nativa recebeu a notícia,
salvo o Temer, aliás, do retorno ao trabalho escravo.

\falaG Sim, o trabalho escravo encaixa bem com a nossa casa"-grande e
senzala. Foi você quem usou pela primeira vez essa terminologia, ou
digamos, analogia entre a casa"-grande e as elites. E a senzala é a
maioria pobre. Essa volta ao trabalho escravo significa um retorno à uma
sociedade escravocrata?

\falaM Não se trata de um retorno porque a casa"-grande, assim como a
senzala, estão de pé. Além de tudo, esse governo de Temer, com a ajuda
do Congresso, e diante de uma alta corte de fancaria, fortalece cada vez
mais a casa"-grande. Estamos falando de um golpe perpetrado pelos Três
Poderes da República. Essa casa"-grande determina uma situação pela qual
só existe uma saída: esse forte abalo social. Não há outra saída. Então
se Lula for condenado em segunda instância o povo tem de sair às ruas
para abortar essa operação. É o único jeito. Não há outro. Caso
contrário nós corremos o risco de não haver eleições no próximo ano. O
próprio Lula disse no final de outubro, durante a caravana pré"-eleitoral
dele em Minas Gerais: ``Eu não sei o que vai acontecer no próximo ano''.
Ele sabe que é esse o risco, ou seja, o de um golpe dentro do golpe para
não haver eleições presidenciais. É hora de reagir.

\falaG Suponhamos que o povo não reaja e Lula seja inelegível em outubro de
2018. Lula, não há a menor sombra de dúvida, é um excelente cabo
eleitoral. É muito provável que esse candidato petista vença o pleito
presidencial.

\falaM O candidato que Lula escolher, o Lula da vez, parte favorito no meu
entendimento. Mas repito: a vitória de um petista apoiado por Lula no
pleito será fatal. Portanto, no momento da condenação de Lula é preciso
ir às ruas. E os sindicatos? Já tiveram e têm razões de sobra para sair
às ruas. Evidentemente convém um exame de consciência. As lideranças não
chegaram ao trabalhador. Não o doutrinaram, não o conscientizaram. É
fundamental mostrar ao trabalhador qual é a situação real. Foi jogada no
lixo a CLT (Consolidação das Leis de Trabalho; 1943). A questão da
reforma da Previdência está aí. Senhores, o que é isso?

\falaG Acha que esses senhores e o povo entendem o significado de um
impeachment? Em recente entrevista, o embaixador e ex"-chanceler de Lula,
Celso Amorim, lembrou que Richard Nixon não foi destituído, mas
renunciou e o cargo ficou para o Gerald Ford. Pois o Ford continuou a
trilhar o caminho traçado pelo Nixon. Da mesma forma, Bill Clinton não
foi destituído. Mas se tivesse, o vice"-presidente Al Gore teria dado
continuidade ao trabalho realizado por Clinton. No Brasil houve um golpe
camuflado de impeachment, visto que o governo do presidente ilegítimo
optou por outra política econômica.

Mino: Concordo. Mas impeachment houve porque a Dilma foi defenestrada.
Rasgaram a Constituição. Além disso, as chamadas pedaladas fiscais não
seriam um motivo para a destituir porque todos praticavam as tais das
pedaladas. E além de tudo reconheceram que não havia pedaladas. O
Belluzzo foi ao Congresso no dia da votação, ou um dia antes, e disse:
``O que houve foram despedalas''. Despedalas. Os deputados queriam um
pretexto, como o fazem agora para alijar Lula do páreo presidencial. É a
mesma tática.

\falaG Acha que se o Lula tivesse sido chefe da Casa Civil no início do
segundo mandato de Dilma não teria ocorrido o impeachment?

\falaM Certamente não teria ocorrido o impeachment. O Lula, é um mobilizador
das massas, na sua qualidade de único líder popular e nacional
verdadeiro. Há lideranças locais fortes. Mas não nacionais. Na sua
qualidade de líder nacional único, junto à sua qualidade de mobilizar
massas, em um discurso que comove o povo, e também o talento dele como
negociador, ele saberia como negociar a permanência do governo de Dilma
até o fim. Sem impeachment. Eu não tenho a mais pálida sombra de dúvida
a respeito.

\falaG O mensalão foi o primeiro golpe contra Lula e o PT?

\falaM Uma tentativa de golpe, eu diria. Apesar do mensalão o Lula se
reelegeu folgadamente. E não somente. Deixou seu governo com 78 por
cento de aprovação. Portanto, o mensalão não funcionou como golpe. Foi
uma tentativa. Em seguida vem a Lava Jato, uma operação atirada, a
partir da corrupção, no caso da Petrobras, contra o Lula e o PT. Já está
ali está montado o golpe. Ele se concretiza em uma eleição, a de 2014,
em que Dilma ganha apertadamente e é destituída dois anos depois.

G. O ``mensalão'' é história encerrada. Entretanto, a direita continua a
associar o ``mensalão'' ao PT, como se fosse algo de inédito.

M. O ``mensalão'' é o caixa dois. Sempre houve na política brasileira.
Mensalão não é novidade. Mas virou novidade porque o PT tem caixa dois.
Quando me hospedei na Granja do Torto assisti a seguinte cena: A Marisa
dizia: ``Eta vida, ele (Lula) acredita nas pessoas. E Lula dizia:
``Fizeram uma besteira''. Mas até que ponto foi uma besteira se o uso de
caixa dois era absolutamente normal na vida política brasileira? O
problema são regras para o financiamento de partidos na política
brasileira. Agora, o mensalão, assim como o escândalo com a Petrobras,
ocorreu com outros governos anteriores aos de Lula. Não podemos esquecer
da ``privataria tucana''. Fernando Henrique Cardoso tinha um amigo
central, o Sergio Motta. Esse era um operador, como dizem no Brasil,
formidável. O dinheiro que ali circulou só Deus sabe. Foi a maior
roubalheira na história do Brasil.

\falaG Como levar a Lava Jato a sério se o Judiciário está politizado?

\falaM O Judiciário está bastante politizado a serviço da casa"-grande a
ponto de participar ativamente do golpe. E esse Moro e a turma dele
compõe um tribunal do Santo Ofício, visto que toda a argumentação levada
adiante para condenar me lembra muito o tribunal que condenou Joana
D'Arc. É igual. E por trás havia também razões políticas. Como Bernard
Shaw explica brilhantemente em \emph{Saint Joan}, uma de suas
obras"-primas, houve um acordo entre uma parte da Igreja francesa e os
ingleses, até então ocupando uma parte da França. Era contra os ingleses
que Joana d'Arc combatia. Selado o acordo, Joana D'Arc virou uma herege.
Foi queimada viva em 1431. Esse julgamento de Joana d'Arc me lembra
muito o julgamento de Curitiba, onde as convicções substituem as provas.
Estou me referindo à condenação de Lula a nove anos e meio. Nove anos e
meio porque o senhor Moro, maligno, quis lembrar que Lula tem nove dedos
e meio.

\falaG Como obteve essa informação?

\falaM É uma convicção minha. É um sentimento que se abriga no fundo do meu
coração e no fundo da minha mente. E me espanta entre o fígado e a alma.
É um vento a soprar negativamente. Por essas e outras maldades,
arrogâncias, deveria haver uma mobilização do povo, incluindo os
trabalhadores, que está pronto a votar em Lula. Esse povo deveria estar
na rua. Insisto nesse ponto. Não há alternativa. Sem povo na rua para
fazer muito barulho e disposto a um enfrentamento contra a polícia
armada e prepotente, agressiva, não há saída. Repito: a eleição não
garante a manutenção do status quo atual. Portanto, essa turma que já
conduziu toda essa operação até aqui não vai ser tomada por um súbito
impulso democrático por conta própria. Vai melar tudo. Precisamos
entender que a propaganda midiática funciona. Dizem que o PT é formado
por um bando de ladrões. É preciso enfatizar o seguinte: A Lava Jato
serve para condenar o Lula. Este País além de ser ridículo é um País
onde reina a maldade mais desenfreada e praticada impunemente. A
prepotência e a arrogância desses senhores é absolutamente insuportável.

\falaG As primeiras perguntas dirigidas às pessoas que se prestam a fazer as
delações premiadas são sobre o Lula. Indaga\falaM Sabem algo sobre o Lula?

\falaM Querem a pele do Lula. Esse Santo Ofício de Curitiba prestou"-se ao
jogo sujo de uma forma vergonhosa. Mas agora os homens do poder já não
querem a Lava Jato. Gilmar Mendes, o ministro do Supremo Tribunal
Federal (STF), não quer. O Temer não quer. Os parlamentares não querem.
Claro, estão todos incriminados. O Temer, aliás, já está condenado. Vai
ser preso.

\falaG Você diz que um \emph{mea culpa} por parte das esquerdas brasileiras
é fundamental. Inclusive para conscientizar o povo.

\falaM Quem se disse de esquerda no Brasil nunca chegou realmente ao povo
brasileiro. Veja o caso do ex"-trotskista Palocci, que pertencia a famosa
Libelu (Liberdade e Luta; movimento estudantil dos anos 1970). Depois
como ministro da Fazenda no governo Lula se revelou um neoliberal. Na
verdade, não tem um trotskista que não tenha virado um reacionário ao
extremo. Na verdade, é o discurso do Lula que empolga o povo brasileiro.
Tem de ser ele a fazer esse discurso. Os sindicatos, volto a repetir,
têm razões de sobra, montanhas de motivos para fazer muito barulho.

\falaG Espanta o fato de os petroleiros não terem agido como outrora na
privatização da Petrobras.

\falaM Claro que espanta. Isso aconteceu em outros tempos. Tempos distantes.
É hora de um belo exame de consciência e partir para a luta. Mas de
verdade. Nas fábricas. Onde for possível. Tem de fazer muito barulho. E
mostrar aos eleitores que a maioria não quer essa situação. Além de
tudo, somos vítimas, com a exclusão de ricos e super ricos.

\falaG Você fez a primeira capa de Lula em fevereiro de 1978 na semanal
\emph{IstoÉ}. Naqueles tempos houve greves bem organizadas e eficazes.

\falaM O Lula nasceu daquela nova leva de sindicalistas. O PT nasceu ali.
Aqueles sindicalistas sabiam convocar greves. Há pouco tempo fiz uma
palestra na CUT (Central Única dos Trabalhadores) de São Paulo, onde
aliás tenho grandes amigos, e falou"-se em uma greve geral que houve. Um
sindicalista da baixada santista disse: ``Olha eu não sei mesmo se houve
uma greve geral porque até agora me pergunto se os trabalhadores não
saíram de casa por decisão própria ou porque os ônibus não saíram das
garagens''. Essa pergunta me tocou profundamente. Quer dizer, foi uma
greve geral, si\falaM parou praticamente tudo, menos os aeroportos. Mas
realmente fiquei em dúvida sobre a sinceridade daquele movimento. Olha,
repito, não escasseiam razões para fazer greves gerais, e muito mais que
isso, para realmente mostrar o protesto, exibir a queixa, trombetear a
insatisfação. Há motivações às pamparras. Aos magotes. E, no entanto,
você vê, não acontece nada. As pessoas parecem levar uma vida normal.
Portam"-se como se nada tivesse acontecido. Isso é de uma gravidade
extrema. E, insisto, não há saída fora de uma disposição para o embate.
Para o confronto. Sem isso, tudo fica como está.

\falaG Em recente entrevista com Lula, ele me disse que há vários
``Lulinhas'' para substituí"-lo.

\falaM Se ficar claro na eleição que o favorito é um petista indicado por
Lula, e não duvido que isso possa acontecer, haverá, repito, um golpe
dentro do golpe. Não haverá eleição. É isso que temos de entender. E é
por isso que insisto: sem o povo na rua, e é muito povo, todos aqueles
que votam no Lula, não há nada. O problema no Brasil é que a presença da
casa"-grande e até hoje da senzala, é essa permanência da Idade Média sob
o manto norte"-americano. Tal contexto decorre do fato de o Brasil não
ter tido uma guerra de independência. Ao contrário de outros países
Sul-Americanos. O Brasil nunca teve uma verdadeira revolução. Sangue na
calçada correu muito pouco no Brasil.

\falaG Esqueçamos o Dom Pedro I e o suposto grito do Ipiranga, que até hoje
não se sabe se de fato existiu. No entanto, durante a ditadura militar e
das elites, entre 1964 e 1985, morreram 500 pessoas…

\falaM 450.

\falaG Foram torturadas 1,8 milhão de pessoas. Houve uma esquerda no Brasil.
Existem o Movimento Sem Terra (MST), o Movimento dos Trabalhadores Sem
Teto (MTST)…

\falaM De fato, os Sem-Teto e o MST são movimentos reais e souberam
doutrinar a turma deles. Por isso são dois movimentos de esquerda que
chegaram no ponto. Agora, quanto aos 450 mortos quero lembrar que
haveria mais mortos e desaparecidos se isso tivesse sido necessário, aos
olhos da ditadura. No Uruguai, país de pouco mais de 3 milhões de
habitantes, morreram 5 mil. Na Argentina foram 30 mil. No Chile nem
sabemos quantos. Torturadores brasileiros eram mestres de torturadores
em outros países assolados por ditaduras. Somos muito bons em tortura.
Muito bons. O nosso herói é o Duque de Caxias, que comandou o genocídio
do Paraguai, no final do século XIX. E é celebrado em praça pública em
estatuas equestres fantásticas desembainhando a espada. Não tivemos
Bolívar, nem San Martín e O'Higgins na nossa tradição. E por isso somos
medievais até hoje.

\falaG E qual é a solução para isso?

\falaM Temos razões de sobra para finalmente mostrar a cara. As esquerdas
têm de ir para a rua com o povo. Porque no fundo tudo isso mostra o quê?
Tibieza, falta de coragem, medo.

\falaG O PT foi formado em 1980 por um movimento corajoso de esquerda.

\falaM De fato, aquelas greves são a meu ver o movimento de resistência mais
importante a ditadura. Admiro muito a coragem de quem foi para o
Araguaia no final da década de 60, achando que aquilo era a Sierra
Maestra, e 80 mil homens enfrentaram 10 mil soldados. Tiro o chapéu. Mas
aquilo é patético. As greves do ABC não foram patéticas. E ao cabo Lula
foi preso e enquadrado na Lei de Segurança Nacional.

\falaG Preso 31 dias no DOPS (Departamento de Ordem Política e Social;
extinto em 1983).

\falaM Lembro da Vila Euclides cheia até as bordas. Foram momentos
extraordinários. Era o povo na rua. O povo reagia.

\falaG O PT tem quase 20 por cento do eleitorado. Lula concorda ser maior
que o PT. Da mesma forma, diz o ex"-presidente brasileiro, Jacques Chirac
era maior que sua sigla de centro"-direita. Willy Brandt foi maior que a
social"-democracia. Em contrapartida, Lula diz que François Hollande foi
menor que o Partido Socialista No entanto, Lula ressalva que chegou a
Presidência sobre os ombros do PT.

\falaM O Lula tem razão. Mas o PT no poder, confesso, foi uma grande
decepção. Por quê? Acreditou em conciliação com os neoliberais.

\falaG Creio que Lula tentou ser pragmático. Quis mostrar às elites que não
era um homem perigoso. E agora ela está mostrando a sua face.

\falaM Mas isso foi perda de tempo. Sabe por quê? Porque os ricos e super
ricos são uma minoria. O povo brasileiro é a maioria. Os pobres são a
maioria. Em lugar de morar debaixo das pontes eles devem partir para as
calçadas.

\falaG Houve também a espinhosa aliança do PT com o PMDB.

\falaM Mas aí há uma injunção política. É preciso entender que se você
quiser governar e ter uma maioria no Congresso evidentemente você tem de
ter um aliado. E, no caso, a sigla sempre disponível é PMDB, o partido
do poder. A sigla que se move conforme as circunstâncias. É normal
haver, portanto, um entendimento com o PMDB. O ponto, a meu ver, é
outro. Houve demasiadas concessões. E agora Lula deveria parar de falar
bem do Meirelles na sua pré"-campanha. Esses tipos de figuras são
tristes, não funcionam. Muitas orientações da política econômica dos
governos Lula e depois dos governos Dilma foram neoliberais. Lula
colocou o Meirelles, porta"-voz do neoliberalismo nativo, no Banco
Central, e o Mantega no seu posto de ministro até o fim…

\falaG Por vezes o Lula se cercou de ministros preciosos como o chanceler
Celso Amorim. Ao mesmo tempo, nomeou gente como o Meirelles, o Mantega,
o Palocci. Como explicar isso?

Mino: Lula foi às vezes ingênuo. O caso do Palocci, o ex"-trotskista da
Libelu que virou neoliberal, é um exemplo. Palocci era o amado prefeito
de Ribeirão Preto, onde já tinha feito várias porcarias. Por exemplo,
privatizou até a água da cidade. A política dele não foi antineoliberal.
Evidentemente o Lula a aprovou. Palocci não poderia implementar uma
política neoliberal sem a aprovação do Lula. Portanto, Palocci se dispôs
tranquilamente a ir adiante. Lula preocupava"-se com a política social e,
por tabela, veio o Bolsa Família, a abertura do crédito, Minha Casa,
Minha Vida, e várias outras iniciativas desse tipo. No entanto, a
política econômica do País foi marcada por uma tendência no mínimo
neoliberal. E além do Palocci, outros neoliberais como o Meirelles e o
Mantega não podem ser esquecidos.

\falaG Mas ao contrário de seus antecessores, Lula implementou programas
sociais importantes para tirar milhões da miséria.

\falaM Claro. Como eu disse, houve avanços sociais importantes. Mas nem por
isso deixou de existir nos seus governos uma orientação básica
neoliberal. Isso me parece negativo em uma análise fria. E está na
origem da atual inércia do PT. Não houve a necessária operação no
sentido de conscientizar o povo brasileiro. Foi a ocasião em que isso
deveria ter sido feito. Não acho que a conscientização do povo se
resolva simplesmente com o Bolsa Família e abrindo o crédito para os
mais pobres. Foram programas sociais importantes, claro. Mas qual foi a
conquista em termos de cidadania de quem estava melhorando de vida?

\falaG A conscientização do povo e os programas sociais deveriam ter sido um
processo concomitante…

\falaM Não há dúvida. Veja o resultado. Sim, houve brasileiros que entraram
no ciclo do consumo. Fala"-se em 35, 40 milhões. Mas os ricos e super
ricos ficaram mais ricos e super ricos. Permaneceu a distância abissal
que marca profundamente a monstruosa desigualdade neste País. E os
coitados dos pobres já voltaram às condições de antes. Aqueles que
tinham progredido já voltaram para trás.

\falaG Mas isso por conta das reformas feitas pelo atual governo ilegítimo
nos últimos 15 meses.

\falaM Estamos diante de um desastre. As reformas atingem os trabalhadores.
Existe uma clara entrega do Brasil ao capital estrangeiro. Enquanto um
imbecil chamado Vargas Llosa, tão endeusado, coitadinho, diz que o
nacionalismo é o mal dos povos. O que ele quer, ser Jesus Cristo? Cristo
morreu na cruz porque ao contrário de Vargas Llosa defendia a ideia da
igualdade. Quem é pela igualdade está disposto a enfrentar todos os
perigos, todas as ameaças.

\falaG O Lula disse em outubro, durante a pré"-campanha dele em Minas Gerais,
que se eleito revogará, através de um referendo, as reformas do PEC 55
que o povo julgar inaceitáveis realizadas durante o governo ilegítimo de
Temer.

\falaM Lula seria um presidente que inaugura novamente uma estação
democrática. Os dois mandatos como presidente dele foram democráticos.
Assim como foram os da presidenta Dilma Rousseff, até ela ser
destituída. Nos governos petistas, quero sublinhar, nem tudo a meu ver
foi acerto porque implementaram políticas econômicas neoliberais. Mas de
qualquer maneira houve avanços sociais e ainda mais importante, a meu
ver, uma política exterior excelente, de grade independência do Brasil.
E sem forçar a barra inutilmente. Uma política exterior muito sutil e
muito inteligente.

\falaG O balanço dos governos do PT é positivo, o que explica o nível de
aprovação de Lula 78\%. O PT baixou o nível de desemprego para 4,3\%.
Durante 12 anos de PT no governo houve um aumento real de salários para
os trabalhadores acima da inflação. Foram gerados 22 milhões de
empregos. Enquanto isso, a Europa gerou 100 milhões de desempregados; 40
milhões saíram da miséria. De que forma a conciliação de Lula com os
neoliberais afetou esses dados?

\falaM Houve dados da política econômica do Lula positivos. Por isso digo
que o Lula foi benéfico do ponto de vista social. Entretanto, também o
mercado se enriqueceu incrivelmente. Os ricos e super ricos ficaram
ainda mais ricos. Recentes pesquisas mostram que o abismo entre uns e os
outros continua igual.

\falaG Este abismo se deve, entre outros, a falta de uma reforma tributária
e de uma reforma agrária.

\falaM Nada de realmente profundo foi feito. Taxar os ricos, por exemplo.
Nem passa pela cabeça de qualquer governo brasileiro. Lula taxou? Dilma
taxou? Não.

\falaG No entanto, Lula não tinha maioria no Congresso para fazer certas
reformas. Foi o caso da regulamentação da mídia. Na França, por exemplo,
o leque de jornais de todas as ideologias existe graças às subvenções do
governo. Lula me disse que fez um congresso para regulamentar a mídia.
Só não compareceram a Globo e a Record. E como ele não tinha maioria no
Congresso e estava em 2010, final de mandato, passou o caso para o
primeiro governo da Dilma.

\falaM Mas se o pais é democrático você terá inevitavelmente uma mídia a
representar diferentes tendências políticas. O Brasil é o País da
casa"-grande. Aqui há um monopólio disfarçado em oligopólio, sinal de que
existe a casa"-grande e a senzala. Somos totalmente medievais. Como você
regulamenta?

\falaG Uma empresa não pode ter um número X de plataformas midiáticas como
jornais, estações de rádio, canais de televisão etc.

\falaM Mas os petistas gostavam era da Globo. Pelo amor de Deus. Adoravam
aparecer na Globo. A revista \emph{Exame} tinha mais anúncios do que a
revista \emph{CartaCapital}. A \emph{Exame} é quinzenal e dirigida ao
baixo clero das empresas. A \emph{CartaCapital} e uma semanal de
política, economia e cultura dirigida para um público geral. A
\emph{Exame} tinha mais anúncios! É para rir. Aqui não há diários,
semanários e jornais de televisão da grande mídia que não sejam
conservadores ou reacionários. A imprensa nanica de esquerda inexiste.

\falaG No Brasil você lê um jornal ou assiste a um noticiário televisivo e é
como se tivesse lidos todos os jornais e visto todos os noticiários de
TV. Enquanto isso, as pessoas continuam levando vida normal, como se não
vivessem em um Estado de exceção. Na França, na Espanha e na Itália já
estariam nas ruas por muito menos do que ter um presidente ilegítimo
como o Temer e um Estado de exceções, como diz o Pedro Serrano. Uma
vasta maioria dos brasileiros parece não se dar conta que não vive em
uma democracia.

\falaM Um País com tamanha desigualdade não pode ser democrático. Temos de
entender isso de uma vez por todas. E tem uma imprensa alinhada toda de
um só lado. Que democracia é esta? Onde existe isso? Vivemos na Idade
Média até hoje. A casa"-grande está aí. E a senzala é visível a olho nu.
Vivemos em um regime de exceção vestido de ilegalidade. A Argentina, ao
contrário do Brasil, é uma democracia. O motivo? Tem uma sociedade muito
mais equilibrada. Comportou"-se desde o século XIX de outra forma com a
presença, entre outros, de San Martín, como já dissemos. Houve uma
verdadeira guerra de independência lá. No século XX, houve uma enorme
reação nas ruas contra as ditaduras. Houve sangue na calcadas. No
Brasil, ditadores são nomes de viadutos. De pontes. É um negócio
inacreditável a nossa leniência. A Lei de Anistia foi promulgada pelo
Figueiredo em agosto de 1979. Fizeram uma Comissão da Verdade. Anistia
para quem? Para os torturadores. Episódio lamentável. Vergonhoso.
Portanto, aqui ainda lutamos contra a casa"-grande e a senzala. Lutamos
contra essa desigualdade monstruosa. A desigualdade no Brasil gera
violência. No ano passado, foram mortas mais de 60 mil pessoas. Sete
assassínios por hora. Este País pode ser democrático? E os ricos e super
ricos levantam muralhas em torno de suas vivendas. Vivem em condomínios
fechados. Andam de helicóptero com medo de serem sequestrados e para
evitar o trânsito. Seus carros são blindados. Assim como proliferam os
valet parking no Brasil, existe uma caterva de seguranças engravatados
enquanto os patrões deles andam de bermudas. É um País ridículo. Somos
ridículos. O Bobbio define admiravelmente essa questão. Ou seja, quem é
a favor da igualdade é de esquerda. Mas você tem de ser sincero nessa
sua escolha. Não pode ser um palavrório inútil. Tem de ser algo muito
bem definido. No Brasil quem é a favor da igualdade com absoluta
clareza? O movimento sem teto, o MST, algumas pessoas certamente. Mas a
esquerda brasileira não existe. O PT nasceu como um partido de esquerda.
Mas no poder portou"-se como? Um partido de esquerda?

\falaG Mas agora o Lula fala em conquistar a casa"-grande. O discurso dele e
do PT deu uma guinada para esquerda. Parece que o PT aprendeu a lição.

\falaM Discordo. O PT não aprendeu a lição. E não produziu grandes
lideranças. Algumas figuras se esforçam. O Lidenbergh Farias, por
exemplo. O Requião não é petista, mas faz o diabo. São figuras isoladas.

\falaG Nos \emph{Cadernos do Cárcere}, Gramsci fala no impacto no povo por
parte de acadêmicos, intelectuais e jornalistas via faculdades, textos,
cinema, rádio e da imprensa escrita. Ironia das ironias, no Brasil essa
revolução passiva para atingir ideais hegemônicos igualitários, no
entanto, parece ter sido realizada pelos neoliberais. O Brasil tem
intelectuais capazes de inverter o quadro?

\falaM Há vários. O Bosi, por exemplo. O almirante Othon, preso pela Lava
Jato. Figura de grande peso internacional. O Fiori é um cara agudo,
ótimo, sobretudo quando se trata de política internacional. O consenso
de Washington foi feito contra ele. Eis um intelectual muito capaz e
ativo. O Wanderley dos Santos é outro intelectual muito envolvido. O
Roberto Amaral, o Celso Amorim. O Pedro Serrano é o mais arguto
definidor desse golpe e as suas consequências. Sem falar do Boulos, que
é realmente de esquerda e faz um bom trabalho. Dentro do PT, eu já
disse, tem o Lindenbergh Farias. O Requião, do PMDB, é um lutador
infatigável. E existem vários outros na Facamp, na Unicamp e na USP,
entre outras universidades.

\falaG No entanto, eles não conseguem fazer esse trabalho de
conscientização.

\falaM Creio que conscientização é um verbo de outros tempos. Queremos que
os leitores falem com seus parentes, seus amigos etc. E diga\falaM ``Olha
gente, vamos botar a cabeça no lugar''. Mas, no caso específico dos
conscientizadores, eu acho que aí a falha foi em grande parte do PT, dos
governos do PT. E de modo geral as falhas, temos de reconhecer, são de
todos aqueles que se disseram de esquerda e não militaram ou tiveram
atuação política em linha com sua ideologia.

\falaG O Lula me falou em nossa entrevista que há um declínio da política. A
meu ver isso ocorre devido a ditadura do capital ter afastado a
política, políticos e partidos do povo. Isso explicaria movimentos como
``Indignai"-vos'', de Stéphane Hessel. Teve forte impacto nos espanhóis
que votaram no Podemos. Houve o movimento contra o mundo financeiro,
Occupy Wall Street. Outra agremiação tentou romper com partidos em sua
maioria caducos: O Syriza, uma aliança de movimentos de esquerda e
ecológicos liderada por Tsipras, na Grécia. Nós concordamos que antes da
forte possibilidade de o Lula perder em segunda instância, seria
necessário um movimento nas ruas. Repito uma pergunta anterior, visto
que repetir é crucial para que as pessoas entendam, como dizia o
pensador e jornalista radical do século XIX Mazzini. Esse movimento no
Brasil seria liderado pelo PT, sindicatos, MST, Sem-Teto, e outras
siglas progressistas?

Mino: Creio que antes da sentença pela segunda instância, seria
importante que houvesse manifestações. Dia 12 de novembro, quando entrou
em vigor a nova Lei Trabalhista com todas as suas maldades e
malignidades, teria sido uma grande ocasião para os trabalhadores saírem
nas ruas e fazerem um barulho do capeta. O Lula é uma liderança forte.
Segundo uma pesquisa do dia 14 na Vox Populi, o Lula passou os 40 por
cento de aprovação. Isso mostra o poder dele. Acho que cabe a ele
engajar o povo. Em primeiro lugar, os trabalhadores, isso com o apoio
dos sindicatos, que não lhe faltaria. Por ora, os trabalhadores estão
quietos. Esse País é estranho. Existe o povo, mas existe também o povão
para estabelecer a diferença.

\falaG Qual é a diferença?

\falaM O povo é a nação, que teoricamente não existe. O povão são os
coitados. São a malta infecta. Rude e ignara. Esse é o conceito. Mas de
qualquer maneira, o povo tem de fazer uma pressão tamanha para assustar
essa gente. Alguns episódios novos estão acontecendo. Por exemplo, o
movimento tomataço. Atiraram às pamparras tomates no ministro Gilmar
Mendes, que foi assistir um jogo de futebol. Mendes aprovou, entre
outras coisas, essa concessão notável aos ruralistas: o trabalho
escravo. Isso na verdade interessava em primeiro lugar ao Temer, que
comprou os votos de deputados da bancada ruralista para que votassem na
permanência dele no poder na segunda votação no Congresso. E assim,
Temer evitou uma segunda tentativa para que fosse destituído.

\falaG O Gilmar Mendes está te processando.

\falaM O Gilmar Mendes me processa na primeira instância. Se por acaso eu
for condenado recorrerei. Ele me processa por causa de um editorial em
que o chamo de ``besta''. A meu ver, em um país civilizado e democrático
um jornalista, pode dizer que o cara é uma besta. Agora, o Gilmar é o
Darth Vader brasileiro. Quando põe aquela capa preta de juiz ele encarna
o figurino na perfeição. À juíza que me interrogava eu disse isso. Ela
começou a rir. Há magistrados bem"-humorados no Brasil.

\falaG Como explicar essa comparação entre a Lava Jato à operação italiana
\emph{Mani Pulite} (Mãos Limpas)?

\falaM O grande cara era o Borelli, chefe da força tarefa. O magistrado
Colombo disse: ``Se nós tivéssemos feito aquilo feito por vocês na
Itália seríamos nós, magistrados, que terminaríamos na cadeia''.
Textual. E afirmou ter deixado de ser juiz porque percebeu o seguinte: O
combate à corrupção não funciona com operações como \emph{Mani Pulite}.
Não funciona. Hoje a política italiana está tão corrupta quanto era
antes da operação Mãos Limpas.

\falaG Mas houve um peso pesado como o neosocialista Bettino Craxi que fugiu
para a sua villa na Tunísia. \emph{Protégé} de Craxi, o Berlusconi se
candidatou a premier para conseguir imunidade política no início dos
anos 1990. Mãos Limpas teve um impacto tremendo.

\falaM Houve uma mudança brutal. Mas o Colombo disse aqui no Brasil que
essas operações não funcionam. É a opinião dele. De fato, manter pessoas
na cadeia anos a fio é uma forma brutal de tortura. É uma coisa
inconcebível do ponto de vista jurídico. Um preso tipo Marcelo Odebrecht
entrega a mãe dele. Diz que a culpa e toda da mãe em uma delação
premiada.

\falaG O doleiro Funaro diz que ele pegou dinheiro do Cunha para oferecer
propinas no Congresso com o fim de apoiar o impeachment de Dilma. Essa
não foi uma delação importante?

\falaM Claro. Mas o Funaro foi preso faz pouco tempo e já se prestou a
delatar. A prisão de gente da Odebrecht, por exemplo, é diferente. Estes
são presos por anos e anos com o objetivo de entregarem até a mãe deles.
Isso é tortura. Mas veja: a delação premiada pode funcionar no caso do
terrorismo. Na Itália, os \emph{pentiti} (arrependidos) mafiosos
delataram outros mafiosos e funcionou muito bem. Mas o \emph{pentito}
que delata é diferente do Marcelo Odebrecht. E agora os atuais senhores
do poder, ou melhor, as quadrilhas no poder, não querem a Lava Jato. Só
querem é a pele de Lula.

G Lula diz que quando assumiu o poder, em 2002, a bolsa de valores de
São Paulo tinha 11 mil pontos. Quando ele deixou o poder ela tinha 71
mil pontos. Ele não sabe se agora as elites estão com raiva dele ``por
razões ideológicas ou se é uma questão de pele''. Por que esse ódio em
relação ao Lula por parte dessa elite empresarial e financeira?

\falaM O ódio em relação ao Lula existe em primeiro lugar devido a uma
pressão norte"-americana e é puramente ideológica.
Por quê? Porque o Lula é
considerado um líder perigoso que representa um certo tipo de esquerda
que não agrada aos Estados Unidos. Esse é o ponto inicial. Por quê houve
o golpe de 1964? Já então os Estados Unidos tinham aversão à esquerda
brasileira. No Brasil o golpe contra o Lula também foi motivado por essa
questão ideológica. Mas também por uma questão de pele. E o ódio de
classe. Lula representa quem? A senzala. Aos olhos dessa gentalha o Lula
é a senzala. É a figura central da senzala. Figura que tem esse poder de
ganhar eleições como um candidato imbatível. E tem o poder de convocar
as massas. Ele mobiliza. Isso é apavorante. Por essas e outras,
eliminemos o Lula para o pessoal da casa"-grande ficar contente. A
começar pelos Estados Unidos. O que querem os norte"-americanos? Ser
súditos. Querem entregar o País oferecendo"-o a preço de banana ao
capital estrangeiro. Ou seja, o Brasil adota uma postura entreguista. E
os Estados Unidos são os súditos, os nossos protetores.

\falaG Você falou em ódio de classes.

\falaM No Brasil o preconceito de classe é brutal. Temos Sowetos espalhados
por todo o País. Nas cotas de ingresso nas faculdades, as escolas bem
pagas, a educação no Brasil é uma vergonha. Quem acaba na cadeia no
Brasil são em primeiro lugar os negros. Depois pobres. Coloque, por
exemplo, o Ronaldão, tão amado, inclusive pelos ricos, que o veem como
um negro de alma branca, à meia"-noite na esquina do Colégio Dante
Alighieri com a Peixoto Gomide, em São Paulo. Passa a Rota da Polícia
Militar. Eles o pegam e o atiram dentro da perua. Não tem erro.

\falaG Esse racismo envolve a classe social e a cor do outro.

\falaM Aqui existem duas formas de racismo gravíssimas e claríssimas. Uma é
o racismo racial. O outro é o racismo social. Estão entrelaçados. Mas,
digamos, o coitado branco também está estrepado devido ao racismo
social.

\falaG Esse sadismo das elites, escreve Gilberto Freyre, remonta à crueldade
dos filhos dos donos da casa"-grande que brincavam com inaudita violência
com os moleques, os filhos dos negros da senzala. Os meninos brancos
usavam os moleques como cavalos, davam chicotadas neles e por aí vai.
Freyre escreve: ``Não há brasileiro de classe mais elevada, mesmo
nascido e criado depois de oficialmente abolida a escravidão, que não se
sinta aparentado do menino Brás Cubas na malvadeza e no gosto de judiar
com negro. Freyre fala em ``deleite mórbido''. Isso explicaria esse
atual ódio pelo povo.

\falaM Acho perfeito. A esquerda brasileira execrou o Gilberto Freyre porque
era um liberal conservador à moda antiga nas suas posturas ideológicas.
E veja que descrição perfeita. E as esquerdas são contra as duas
obras"-primas do Gilberto Freyre.

\falaG Você prefere \emph{Sobrados e Mocambos}.

\falaM Acho ótimos os dois livros. Nós também temos os sobrados e mocambos,
os castelos e as favelas.

\falaG Concorda que as elites têm mais consciência de classe do que o povo?

\falaM O povo brasileiro não tem consciência de classe. Um pouco, talvez, em
certos níveis. No fundo, não existe preconceito racial e social por
parte do povo. Socialmente são todos iguais, mais ou menos. Agora, eu me
pergunto. Os avanços sociais conseguidos levaram à formação de um
cidadão beneficiado? Criaram a consciência da cidadania? O sujeito
beneficiado não tem noção da sua condição de cidadão. Sua ideia da
cidadania não é formada. Ele votará provavelmente em Lula porque foi
beneficiado. Mas isso não significa que ele tenha consciência da
cidadania. O que caracteriza um povo é a consciência da cidadania de
todos. O brasileiro não é um cidadão.

\falaG Segundo Lula, a chave é o mercado interno para que o povo possa
consumir.

\falaM O Lula tem razão em acreditar no mercado interno. Esse povo
espezinhado é um tesouro. Se você permite que o povo evolua você
eliminará o abismo entre pobres e ricos e terá um bom consumidor. Isso
será a forca do País. Agora, uma política agrícola é muito menos
importante do que uma política industrial. De qualquer maneira. Sem
contar que o Brasil exporta soja e o minério de ferro. E os preços desse
tipo de commodities caíram. E agora, como exportador de petróleo, o
Brasil vai deixar de ganhar o que poderia. Privatizada a Petrobras, essa
receita irá à Exxon e a outras empresas estrangeiras. O que interessa é
uma política industrial. Mas a indústria brasileira foi destruída. Esse
País, a décima quinta potência industrial do mundo a certa altura de sua
história, e durante uns trinta anos, foi liquidado. Getúlio Vargas se
suicidou em 1954 porque queriam que ele fosse embora. Mas não deu certo
porque em seguida veio outro desenvolvimentista, o Juscelino Kubitschek.
Em seguida, o Jânio renunciou. Aí veio o golpe que derrubou João Goulart
em 1964. O Jango realmente tinha planos para uma reforma de base que
puseram os ricos em polvorosa.

\falaG Com o golpe de 1937, Getúlio criou uma infraestrutura para o
desenvolvimento industrial. Isso é fundamental para a criação de uma
economia forte, como o sabem bem os lideres alemães, italianos e
franceses, entre outros.

\falaM A industrialização foi o caminho que ele buscou e era na época o
caminho para o desenvolvimento. Dentro desse conceito ele criou em
primeiro lugar Volta Redonda, que produz aço. Depois consolidou as leis
do trabalho. Depois criou o salário mínimo, que à época era muito
superior ao salário mínimo atual. Em suma, ele criou as bases efetivas
para uma industrialização. Essas bases serviram realmente para que o
Brasil se tornasse entre Getúlio e Jango a décima quinta potência
industrial do mundo. De longe a maior da América Latina. Foi um período
importante. Getúlio depois caiu, mas voltou. E procedeu no caminho de 51
a 54. Criou a Petrobras em 1952.

\falaG Lembro de uma entrevista que fiz com o Abílio Diniz em 2012 na qual
ele dizia maravilhas sobre o Lula e a Dilma. E agora leio que o Diniz, o
Armínio Fraga, o Nizan e o Huck criaram o ``Foro Cívico'' para apoiar
candidatos.

\falaM Os candidatos confiáveis a eles.

\falaG Descartar o Lula por um governo ilegítimo e apoiar um novo candidato
neoliberal é uma inenarrável forma de oportunismo. Diniz badalou o Lula
no poder e depois quando não está no poder se junta a esses empresários
para inventar o candidato deles.

\falaM Trata"-se de gente que não sabe reconhecer os méritos do Lula. O
Armínio Fraga não coloco nessa equação de gente vira"-casaca. Sempre foi
um neoliberal. No entanto, quando ainda estava no Pão de Açúcar, o
Abílio ganhou dinheiro como nunca na época do Lula porque havia abertura
de crédito para gente que antes não gastava começou a gastar, inclusive
nos supermercados dele. E ele ficava entusiasmado. Hoje essa gente não
quer reconhecer os méritos do Lula. Aceitaram o Lula porque era o homem
da vez. E porque o Lula é também um homem cativante, charmoso. O carisma
do Lula pode ser muito sedutor.

\falaG Temos também o Scaff, da FIESP. Ele fala em estabilidade da economia
sob Temer.

\falaM Esse Scaff nem é empresário. Trata"-se de uma figura terrível,
caricata, entre outras coisas. Faz parte de uma gente com rostos
patibulares, expressões trágicas. Esses são as pessoas que nos cercam no
poder. Veja as sessões no Congresso ou no Supremo Tribunal Federal. As
reuniões dos supostos guardiões da lei. É impressionante a pobreza. A
pobreza intelectual, cultural, a mediocridade primitiva das pessoas. E
todos certos de sua importância. Todos incapazes de qualquer lance de
autoironia. Grotescos. Cafajestes. Primários. Primitivos.

\falaG Segundo Lula, a direita está tentando criar um novo candidato, visto
que não há nenhum.

\falaM Essa tentativa existe. A questão é a seguinte. Os candidatos existem
em função de Lula. Então hoje, nesse momento, o candidato mais forte
depois de Lula é o Bolsonaro. O Bolsonaro tem menos da metade de
intenções de voto do que Lula. Mas ele é o mais forte na oposição. Por
isso digo que não haverá eleições se o Lula vencer. A turma no poder não
vai achar um candidato para disputar o pleito contra o Lula.

\falaG É difícil compreender a ascensão da extrema"-direita no Brasil. O
deputado Jair Bolsonaro promete maior segurança, quer armar o povo.

\falaM Te pergunto: Como explicar o movimento evangélico no Brasil?

\falaG O movimento pentecostal apoia o Bolsonaro.

\falaM Bolsonaro é um retrógrado. Primitivo. Se ele fosse presidente,
instituiria o ensino do criacionismo nas escolas. No tempo da ditadura
ensinavam moral e cívica. Teremos o criacionismo. É um delírio absoluto.
A recente exposição de quadros onde aparecem nus no Masp em que foi
proibida a entrada de menores de 18 anos foi uma aberração. É como se
colocassem na porta da Capela Sistina, em Roma, onde há, entre outros,
um Adão nu, uma placa proibindo a entrada de menores de 18 anos.
Crianças não podem ver o Adão nu.

\falaG Essa mentalidade cultural explicaria a ascensão de Bolsonaro?

\falaM Falam no fascismo de Bolsonaro. Mas não é por aí. Bolsonaro
representa antes de mais nada uma espécie de facínora eleito para ser
deputado. Toma as posições mais retrogradas que se possam imaginar.
Certamente acredita no criacionismo. Ele não existe sem o Lula. O Lula
vai ser o pior entrave de quem está no poder.

\falaG E os outros candidatos?

\falaM Não existem. O Doria não passa de um coitado. Um marqueteiro. Não dá
para levá"-lo a sério. Demonstra o nível em que se encontra esse País. E
estão cogitando até Luciano Huck. Dá para acreditar? Veja ele em ação. É
penoso. O Alkmin, como candidato a presidência é um perdedor. Conseguiu
ser governador. Os tucanos estão em São Paulo há 24 anos.

\falaG Esse reacionarismo de São Paulo remonta à chamada Revolução de 1932?

\falaM Remonta ao fato de que era a locomotiva, em primeiro lugar. A
locomotiva carrega o comboio. Leva o comboio a se arrastar sobre os
trilhos. No início do século XX os cafeicultores, os comissários do
café, eram os donos de São Paulo. Era uma turma brutal. À época houve
uma leva imigratória bastante positiva. As greves de 1907, 1909, 1917
foram importantes. Os organizadores foram 400 anarquistas. Altino
Arantes, já governador de São Paulo, os deportou. Mandou"-os de volta
para a Itália. Outra imigração positiva como a japonesa não representava
entrave político. Com os anarquistas o operariado começou a se
manifestar. Mas quando foram deportados o reacionarismo afundou as suas
raízes aqui de forma extraordinária. Culminou com a Revolução de 1932.
Eles chamam de revolução, mas era um movimento separatista. O Brasil não
estava, porém, maduro para uma guerra de secessão.

\falaG Esse ódio ao imigrante por parte dos chamados ``quatrocentões'' não
pode ter sido projetado contra o nordestino, e no caso contra Lula?

\falaM Durou muito tempo esse ódio ao imigrante. Mas contra o Lula,
nordestino, diminuiu. Tomou outra forma. Encaixou"-se no ódio de classe.
Antes, essa repulsa ao nordestino foi muito mais forte. Não houve
políticas para segurar os nordestinos nas terras deles. Nunca houve. E
teria sido importante oferecer a eles condições melhores de vida e de
desenvolvimento.

\falaG Do ponto de vista jornalístico, é importante entender como
funcionaria um movimento de reação do povo. A chamada ``Primavera
Árabe'', uma invenção do Ocidente que não resultou em nada significante,
foi organizada via redes sociais. O Donald Trump ganhou a presidencial
bastante ajudado pela Rússia, que teria produzido 80 mil posts que
atingiram 126 milhões de pessoas nas redes sociais, segundo a BBC. No
Brasil, as esquerdas participam muito nas redes sociais.

\falaM Aqueles que acreditam ser de esquerda.

\falaG Há raros esquerdistas. Mas permanecem escudados pelos seus
computadores e celulares. O próprio Lula tem uma participação importante
nas redes sociais. No entanto, os jornalões e revistas reacionárias
ainda dão as cartas. No último ano, disse o Lula, houve 56 capas de
revista contra ele. Como organizar esse movimento de revolta para um
restaurar uma democracia diante do poder desses jornalões?

\falaM Os jornais, de fato, têm tido muito êxito nessa campanha anti-Lula e
anti-PT. Refiro"-me à mídia impressa, mas acho que funciona sobretudo a
Globo. Funciona sobretudo a televisão, que alcança sobretudo os pobres.
Pobre não tem acesso à televisão a cabo. E, portanto, assiste os canais
nacionais. E a televisão também está toda contra Lula. Portanto, Lula
está sujeito à televisão.

\falaG Entre as pessoas mais ricas do Brasil encontram"-se os três Marinho,
filhos do suposto jornalista Roberto Marinho. Teriam entre eles, os três
irmãos, uma fortuna estimada em 11,3 bilhões de dólares. Segundo o
Marcos Coimbra, o Roberto Marinho engoliu o ``sapo barbudo'' achando que
ele não teria folego na política. Não foi o caso. E agora os sucessores
dele querem ``destruir a imagem pessoal e política do ex"-presidente''. O
Lula me disse que no último ano houve 20 horas de jornal da Globo contra
ele. A Globo, sabemos, faz campanha também contra o Temer, mas o que
querem mesmo é a pele de Lula.

\falaM Atacam o Temer por razões morais. Não é possível que o presidente
ladrão continue no cargo. Mas a Globo também faz campanha contra o Temer
porque ele já foi condenado. Não tem nada a ver com o anti-Lulismo
deles. Eles são contra o Lula. Contra Temer e contra o Lula. Vamos
entender claramente. Os Marinho querem colocar no poder um cara mais
ligado a eles. Chega de intermediários.

\falaG É estranho a mídia internacional, salvo raras exceções na Europa,
estar silenciosa a respeito do golpe contra Dilma, o julgamento ilegal
de Lula, e a essa ausência do Brasil no cenário global.

\falaM Em primeiro lugar o Brasil é muito distante. E não é fácil para o
Europeu entender o que está acontecendo. Ironicamente, isso prova que a
casa"-grande está de pé. E isso significa que o Brasil está na Idade
Média. A mídia europeia, como você diz, não entende isso. Veja o quadro:
um golpe perpetrado pela aliança entre Executivo, Legislativo e
Judiciário. E jogando no lixo a Constituição de 1988 que, certa ou
errada, perfeita ou não, completa ou não, é a Constituição. Associe"-se a
tudo isso uma mídia completamente alinhada contra o Lula e contra o PT,
e, portanto, apoiando os Três Poderes da República. E, ainda, há setores
da polícia transformados em jagunços da casa"-grande. Explique isso para
um Europeu. Ele não tem condições de entender uma situação dessas.

\falaG \emph{Financial Times} e o \emph{Wall Street Journal}, por exemplo,
estão felizes com a ``estabilidade econômica'' sob Temer. E, por tabela,
aplaudem a privatização da Petrobras. É a ditadura do capital?

\falaM É a ditadura do capital. O neoliberalismo quer isso. É um desastre
absoluto. E no Brasil esse desastre assume proporções gigantescas. Isso
porque o golpe favorece somente os ricos e os super ricos. Nada além
deles. E esse povo espezinhado precisa sair às ruas. A conciliação na
qual o Lula já acreditou é impossível. Nós temos de botar isso com
clareza nas nossas cabeças. Se o candidato de Lula ou qualquer outro que
possa alterar o plano golpista engendrado durante o primeiro mandato, a
partir da Lava Jato, haverá um golpe dentro do golpe. Com o apoio dos
Estados Unidos.

\falaG Lula seria ainda o único a poder comandar o País em nível
internacional?

\falaM Quando governou o Brasil o Lula foi uma estrela internacional. Em
2009, a prestigiada revista norte"-americana \emph{Foreign Policy} disse
que o Celso Amorim era o chanceler mais importante do mundo. Acho que a
atual falta de lideranças depende muito dessa ausência da política, que
o Lula com razão aponta. Por conta disso, é difícil que surjam
lideranças se a política sofre. Mas chamo a atenção para lideranças que
estão nascendo no mundo. O Corbyn e uma liderança forte no Reino Unido.
Idem o Mélénchon, na França. O jovem italiano Speranza, de 40 anos, é
uma liderança nascente. Essa situação na qual o mundo todo se sujeita à
força do capital provoca novas lideranças. E surgirão outros e outras
que tomarão a defesa do Estado democrático. Inevitavelmente. O
Varoufakis, ex"-ministro de Tsipras, é o exemplo de um homem que pode
influenciar mudanças econômicas e políticas. Quanto ao Lula, ele colocou
o Brasil no mapa através de uma política internacional independente.

\falaG Trump não gostaria de uma vitória do Lula, mas o Obama e o
ex"-presidente brasileiro se davam bem.

\falaM Se davam bem. Mas a postura muito independente de Lula a era difícil
de ser apreciada também por Obama. A construção dos BRICS e outras
situações não poderiam agradar os Estados Unidos. O Lula, outro exemplo,
foi a Israel e tomou uma posição muito clara pró-Palestina. E ao mesmo
tempo tudo aquilo que se fez em termos de política internacional foi a
meu ver de uma forma muito correta.

\falaG O Celso Amorim me disse em entrevista recente que antes do Lula o
Brasil jogava na segunda liga, e poderia até mesmo ser o primeiro na
segunda liga. Mas sob Lula o objetivo foi passar para a primeira liga. E
fizeram isso. Amorim disse que a política externa de Lula na América
Latina, na África e no Oriente Médio incomodou os países ricos. Entre
outros feitos houve as negociações que aproximaram o Ocidente do Irã, em
2010, capitaneadas pelo Brasil e pela Turquia. Graças a esse pré"-acordo,
copiado depois pelos Estados Unidos, temos hoje negociações entre a
chamada ``comunidade internacional'' e Teerã. Claro, o Trump está
tentando acabar com tudo isso.

\falaM Depois de concluído com sucesso o pré"-acordo, os Estados Unidos não
acharam graça. Mas houve, neste caso, uma carta do Obama respondendo ao
Lula e o autorizando a agir, ou pelo menos louvando a iniciativa. Essa
carta precedeu a ida a Teerã.

\falaG Por sua vez, o Fernando Henrique era submisso internacionalmente.
Adorava os norte"-americanos.

\falaM Completamente. Clinton o abraçava toda hora.

\falaG A voltamos ao poder dos mercados, o que explicaria a submissão de
Fernando Henrique aos países ricos.

\falaM A privatização, quero dizer roubalheira sob Fernando Henrique, foi
aplaudida pelos países da chamada ``comunidade internacional''. Agora
vem aí a privatização da Petrobras. E o que foi a licitação do pré"-sal?
Quando você tira o pré"-sal da Petrobras você diminui o valor da
Petrobras de uma forma brutal. E vão vendê"-la a preço de banana. Aliás,
o Fernando Henrique queria privatizar a Petrobras quando era presidente.

\falaG Objetivo concretizado sob Temer.

\falaM É a grande vitória tardia de Fernando Henrique. É algo assustador. E
por que um professor universitário, aposentado, tem um apartamento em
São Paulo de 1 mil metros quadrados de construção no bairro nobre, como
se diz aqui, de Higienópolis, e uma fazenda de 500 alqueires. Não
estamos falando em um ridículo triplex na praia de farofeiros das
Astúrias. Ou de um sitio em Atibaia com vista para a favela. Os
social"-democratas brasileiros não são os social"-democratas do Willy
Brandt. Eles são o esteio da extrema"-direita, de um reacionarismo feroz.
São a favor, são o instrumento da casa"-grande. Envolvidos, inclusive,
até o talo no golpe. Não há demonstração mais clara.

\falaG Causa estranheza o fato de os críticos de Lula sempre dizerem que
Fernando Henrique colocou ordem econômica no País.

\falaM Esses críticos pensam na estabilidade da moeda. Valendo"-se de alguns
economistas espertos, o Fernando Henrique conseguiu obter uma certa
estabilidade. Na origem o plano de estabilização se inspirou em uma
manobra israelense de muito tempo atrás: a de valorizar a moeda para
torná"-la mais segura. Trata"-se de um mecanismo complexo que alguns
economistas armaram para o Fernando Henrique, que desconfio não entender
nada de economia. Desconfio. De qualquer maneira, o que aconteceu? Já no
governo de Itamar Franco, Fernando Henrique, então ministro do Exterior,
essa manobra israelense foi implementada e a moeda foi estabilizada. Aos
fatos. Durante o primeiro mandato de Fernando Henrique houve o craque
russo e o Brasil quebrou. Para conseguir sua reeleição o Fernando
Henrique comprou votos porque a Constituição tinha de ser modificada
para uma maioria expressiva do Congresso. Conseguida a mudança
constitucional, ele fez uma campanha eleitoral com a bandeira da
estabilidade. O Roberto Marinho acreditava. Nem se fale da Miriam
Leitão. Foi um delírio: estabilidade, estabilidade, estabilidade. Doze
dias após a posse, ou seja, já reeleito, no dia 12 de janeiro de 1999,
Fernando Henrique desvalorizou o real. Medida que abriu um rombo na
empresa da Globo. E o Fernando Henrique Cardoso quebrou o Brasil
novamente. Quando o Lula chegou ao poder, os cofres do Estado estavam
vazios. O Brasil devia mais de 200 bilhões de dólares. O Lula pagou
tudo. E mais: encheu as burras do Estado. Colocou muita grana nos
cofres. O Estado tinha lastro mesmo. Forte. Como podem tecer elogios
sobre a política econômica do Fernando Henrique? Ele quebrou o Brasil. E
a privatização das comunicações foi a maior bandalheira do Brasil. Agora
não sei o que essa turma no poder fará. Tenho a impressão que enquanto a
bandalheira do Fernando Henrique foi muito bem armada, essa me parece de
uma mediocridade lancinante. Eles são ladrões de uma forma diferente.

\falaG Agora a roubalheira é mais transparente. Vemos pessoas correndo com
malas repletas de dinheiro. E de todos os partidos.

\falaM É isso.

\falaG Você falou sobre os policiais federais que se comportam como jagunços
dos poderosos.

Mino: Largos setores da polícia federal aderiram ao golpe de maneira
absolutamente vergonhosa. Apoiaram o impeachment de Dilma e ofereceram
proteção a quem não merece. E estão prontos a se engajar em qualquer
tipo de operação.

\falaG Um militar já se exprimiu em casa maçônica sobre uma possível
intervenção.

\falaM Não acredito em uma intervenção militar. E não sei como se
comportariam se o povo saísse à rua para protestar e mesmo e disposto a
brigar. Não sei como se comportariam em caso de desordem, que de certa
forma prefigura princípio de guerra civil. Essa e uma incógnita muito
séria. Creio que os militares, salvo uma ou outra exceção, têm se
portado muito bem. Por ora, essa é uma força que age dentro da
Constituição. Permanecem fiéis ao papel Constitucional deles. O que não
impede que apoiem as quadrilhas no poder. Os poderosos foram generosos
com os militares. Por exemplo, restabeleceram a justiça militar para
lidar com crimes cometidos por soldados brasileiros. E vale lembrar que
os soldados brasileiros foram enviados ao Rio de Janeiro para realizar
uma operação. O próprio chefe do Exército diz: ``Soldado não é
polícia''. Isso, inclusive, diminui o papel do soldado. De qualquer
modo, se os soldados derem uns tiros mal dados o tribunal militar os
julgará. É uma aberração.

\falaG Por outro lado, os militares podem se voltar contra os poderosos.
Defendem a Petrobras, por exemplo.

\falaM São nacionalistas. Essa história do pré"-sal certamente os irrita
muito. Por ora, respeitam, como eu disse, a Constituição. Quem não a
respeita são os Três Poderes. Juízes da Alta Corte permitiram um
impeachment sem motivação. Permitiram que Lula seja condenado sem
provas. Uma história nefanda. Soltam criminosos.

\falaG No golpe de 1964 houve tanques
para derrubar o João Goulart.

\falaM Em volta de Jango havia esquerdistas sinceros. As chamadas reformas
de base, a plataforma de Goulart, eram coisa séria. Era exatamente taxar
os ricos, reforma agrária em profundidade e assim por diante. Era um
governo de esquerda. E havia gente de esquerda. Não comunistas. Gente de
esquerda no sentido de reformadores do País. Reformadores que queriam
liquidar o imenso abismo que separa ricos e pobres. Queriam acabar com a
casa"-grande e a senzala.

\falaG Washington viu a situação de outro ângulo.

\falaM Os Estados Unidos ofereceram suporte bélico, se precisassem. John
Kennedy, esse gênio da lâmpada, era um desastre. Kennedy é uma história
inventada.

\falaG No golpe de 1964 houve o apoio das elites que derrubaram um governo
de centro"-esquerda, no qual o Estado controlava a economia. No golpe de
2016, sem tanques, o mercado quis dominar o Estado. Esse último golpe me
parece mais sutil.

\falaM O golpe de 2016 foi diferente e adequado ao contexto político no
Brasil e no mundo. Mas veja: o golpe de 1964 foi um golpe desfechado
pelos militares, chamados pelas elites, haja visto os editoriais do
Estadão na época, ou do Globo. Os militares foram chamados pelas elites
para fazer o serviço sujo. Depois talvez os militares tenham gostado
bastante de estar lá em cima. Na verdade, o golpe de 1964 é civil e
militar. Porque a deixa sai das elites. O povo não soube como reagir,
mas começou a medrar uma espécie de resistência que depois se
fortaleceu. Por parte dos ditadores, havia aquela hipocrisia de manter
um sistema eleitoral. Finalmente o AI5, em dezembro de 1968, confirmou
com clareza o golpe. Os militares ganharam então plenos poderes. Com o
golpe dentro do golpe o Congresso foi fechado para reabrir depois de
algum tempo. Houve cassações às pamparras, mas nem por isso a oposição
arrefeceu. E o MDB, comandado pelo Ulysses Guimarães aglutinou todos os
opositores. Alguns partiram para a luta armada e, no caso, morreram mais
de 400 e foram torturados quase 2 milhões. Este é o resultado dessa
tentativa de luta armada. A qual, de resto, os comunistas rejeitaram. O
partidão brasileiro era contra a luta armada. Quem se estrepou lutou
pela volta do regime democrático. De qualquer maneira, a ditadura foi um
episódio que hoje deveria ter fortalecido o anseio pela democracia. Bem,
o que de fato demonstra em primeiro lugar a diferença entre os dois
golpes são os meus sentimentos. Naquele tempo, mesmo perseguido pela
censura, eu tinha grandes esperanças de que chegaríamos finalmente a
reencontrar o caminho, aliás, a encontrá"-lo de uma vez por todas. O
Brasil havia atravessado um período econômico muito favorável de Getúlio
Vargas a Jango Goulart. Foi quando o Brasil chegou a ser a décima quinta
potência industrial do mundo. O que talvez até explique porque acabou
surgindo, já em tempos de ditadura, uma nova geração de líderes
sindicais. De alguma forma estava se criando, ainda que de forma muito
confusa, incerta, um proletariado. O proletariado sempre foi a bucha de
canhão da esquerda. Para mim, foi um período de esperança.

\falaG Inclusive do ponto de vista jornalístico.

\falaM Claro. As coisas estavam entrelaçadas. O golpe e a ditadura que
resultou estranhamente me mostraram a serventia do jornalismo, isto é,
dos jornalões e da Globo. Por isso, senti que as publicações
alternativas poderiam mostrar a utilidade do jornalismo. A época, eu
tinha lido um ensaio da Hanna Harendt sobre a importância do bom
jornalismo honesto. Aquele ligado efetivamente àquilo que podemos chamar
de verdade factual.

\falaG Com o AI5, em 1968, você dirigia a \emph{Veja}. Ou seja, a censura
estava longe de ser uma brincadeira.

\falaM A \emph{Veja} era a maior revista de oposição. Não digeria a
ditadura. Era uma revista postada contra a ditadura. Foi censurada. Mas
te digo uma coisa: essa história da censura e inacreditável nesse País.
É contada de uma maneira realmente singular. Leia os artigos da
\emph{Folha}, do \emph{Estado} e do \emph{Globo}. Narram os tempos de
chumbo. Aonde? Eles nunca foram censurados? O \emph{Jornal do Brasil}
nunca foi censurado. O \emph{Estadão} foi censurado porque os dirigentes
do jornal queriam o Lacerda na presidência. No entanto, o Lacerda acabou
cassado ao criar uma ampla frente com o Kubitschek e o Goulart. Foi
assim que teve início a censura contra o Estadão. A censura era feita na
redação do Estadão e os buracos deixados pelas tesouras censoriais eram
preenchidos com versos de Camões. No \emph{Jornal da Tarde} publicavam
receitas de bolo. O resto não foi censurado. Quem foi censurado
realmente? \emph{Veja} e os alternativos. E censura feroz. Muito feroz.

\falaG Quando o teu entusiasmo pela abertura levou uma ducha fria?

\falaM Quando vi a chamada redemocratização percebi que tinha me enganado
redondamente.

\falaG A partir de 1985?

\falaM Quando Figueiredo sai pela porta dos fundos do Planalto, e caminhamos
para a eleição do Collor.

\falaG Com o apoio da Globo.

\falaM A Globo e a \emph{Veja}, entre outros, apoiaram o ``caçador de
marajás''. Foi então que senti que tinha errado tudo. A minha esperança
estava sem fundamento. Veio então o Sarney. Depois tivemos o Collor. E o
Fernando Henrique Cardoso, uma calamidade. O governo mais corrupto da
história do Brasil. Depois finalmente foi eleito o Lula, em 2002.

\falaG Quando você começou a se interessar pelo Lula?

\falaM Comecei a ouvir falar do Lula porque preparava"-se a greve de 1978. Um
repórter da revista \emph{IstoÉ}, Bernardo Lerner, irmão do David
Lerner, petista da velha guarda que vinha do PTB de Pasqualini e
Brizola. O David Lerner falou bem do Lula para o irmão Bernardo e para
mim. Ele disse: ``Olha, esse cara é muito interessante. Valeria a pena
falar com ele''. Lá fomos nós, o Bernardo Lerner e eu, ao sindicato de
São Bernardo e Diadema. Ao entrar no sindicato logo notei uma reprodução
do quadro do Pelizza da Volpedo, \emph{O Quarto Poder}, título que não
se refere a imprensa, mas aos trabalhadores. O que mostra o quadro? Um
senhor enchapelado, barbudo, cabeludo na frente do povo. E aí o Lula vem
ao meu encontro. Parecia que tinha saído do quadro. Faltava o chapéu.
Descrevi essa cena no meu livro, \emph{O Brasil}. Lula e eu tivemos uma
longa conversa e fiquei muito impressionado. A primeira coisa que
impressiona no Lula é o seu senso de humor, característica muito
importante. Sabe lidar com a ironia. E tem definições muito boas e
transparentes em relação a vários temas. Revelou"-se eloquente sobre a
política naquele momento da ditadura sob o Geisel. Foi uma conversa
muito boa. Falamos sobre a vida dele desde o começo. Desde a viagem no
pau de arara de Garanhuns a São Paulo, sobre a mãe faxineira que aqui em
São Paulo cuidava de oito filhos. Lula falou do único irmão dele que se
interessou por política, o Frei Chico. O comunista. Foi do PCdoB.
Recentemente encontrei com ele. Frei Chico estava saindo do Instituto
Lula e eu chegava. Eu conhecia bem o Chico. Fui à casa dele várias
vezes.

\falaG Segundo o Lula, o Frei Chico não teve um impacto ideológico nele. O
que o irmão fez foi conseguir para ele um posto no sindicato.

\falaM É isso mesmo. Mas de volta ao dia em que conheci o Lula, o Bernardo e
eu decidimos fazer uma entrevista no formato pergunta e resposta com o
Lula. Na introdução descrevemos a chegada dele em São Paulo, e como ele
se tornou torneiro mecânico e chegou ao sindicato. E fizemos a famosa
capa, concebida pelo capista Hélio Almeida e publicada em 9 de fevereiro
de 1978. A greve seria realizada em abril daquele ano porque vencia o
contrato coletivo de trabalho. Foi uma revista do capeta.

\falaG Qual foi a reação à capa?

\falaM A reação dos leitores me pareceu muito boa porque a \emph{IstoÉ} foi
um sucesso de vendas. Ela chegou a 100 mil exemplares de tiragem, em um
tempo em que a Veja tinha tiragem de 200 mil. Quer dizer, era uma
concorrente real. E os ditadores, ao contrário desse atual governo
golpista encabeçado por quadrilheiros, não vetava anúncios. Nesse tempo
a censura tinha terminado. A censura terminou, de fato, no começo de
1977, que foi quando a revista se tornou semanal exatamente por causa
disso.

\falaG E quem foi o homem por trás do fim da censura.

\falaM O general Golbery do Couto e Silva. Ele tinha, de fato, o plano de
retorno à normalidade democrática, segundo o próprio.

\falaG Você me disse que o Golbery facilitou a formação de partidos
políticos como o PT para estilhaçar a esquerda. O Lula não concordou com
essa tese quando fiz a pergunta a ele. Antes de ele ser preso, em 1980,
Lula lembra que já existia uma abertura para formar novos partidos além
do MDB e da ARENA. A histórica dos homens que vinham interrogá"-lo a
pedido do Cacique, o Golbery, existiram de fato. Mas ela não acredita
que foi o Golbery quem facilitou a formação do PT.

\falaM Eu nunca disse que o Golbery facilitou a formação do PT. Disse que o
Golbery estava muito interessado na personagem. E entendia que o Lula
era diferente dos pelegos. Tratava"-se de um sindicalista de QI alto.
Para o Golbery, o Lula tinha um pensamento não marcado pelo marxismo,
não marcado por uma postura esquerdista tradicional, digamos. Era
esquerdista no sentido que ele queria a igualdade. Queria que os
trabalhadores fossem tratados da melhor maneira possível. Golbery sentia
em Lula a pessoa sem intenções de ser um esquerdista feroz e bem
embasado. Só isso. O Golbery, repito, não facilitou a formação do PT. Eu
nunca disse isso. Golbery estava muito interessado em fazer várias
perguntas ao Lula. Ele sabia que eu era muito amigo do Lula. Fazíamos
várias capas da \emph{IstoÉ} sobre o Lula. A uma certa altura o Golbery
perguntou: ``Como é esse Lula?'' Foi então que mandou gente para ouvir o
Lula no DOPS. E perguntavam o que esses emissários do Golbery?
Perguntavam ao Lula como via a vida, o que pensava do mundo, como via o
trabalho no Brasil, o que achava dos empresários. Perguntas desse tipo.

\falaG Mas houve, de fato, uma verdadeira tentativa do Golbery de estilhaçar
as esquerdas como fez o Mitterrand com as direitas na França nos anos
1980.

\falaM No caso do Brasil não foi uma tentativa para estilhaçar as esquerdas.
O Tancredo Neves e o próprio Ulysses Guimarães eram conservadores
inomináveis. O Tancredo, embora subisse nos palanques das ``diretas
já'', não queria as ``diretas já''. Queria eleições indiretas.

\falaG E o Ulysses Guimarães queria as diretas.

\falaM E ganharia. Foi o senhor diretas. Mas não era um homem de esquerda.
Absolutamente. Havia ao redor dele paspalhos como o Fernando Henrique
Cardoso, que se dizia de esquerda. Serra era a mesma coisa. E estavam
todos no MDB. Inclusive o PT, também no MDB, votaria no Ulysses
Guimarães se houvesse diretas sem reforma partidária, ou seja, antes da
reforma partidária. O Montoro também fazia parte desse grupo no MDB.
Ótima pessoa. Era um democrata"-cristão a velha moda. A turma da oposição
estava toda no MDB do Ulysses. Vamos deixar bem claro: não era uma
esquerda. Havia uma minoria de esquerda, inclusive marxistas. Tratava"-se
de uma oposição valente. Donde a ditadura ter recorrido a inúmeras
cassações. Deixava o Congresso aberto, mas caçava os congressistas
quando bem entendesse. Houve o famoso pacote de abril, em abril de 1977,
começo da \emph{IstoÉ}. Com o pacote de abril fizeram um rapa na
oposição.

\falaG Como se deu o teu contato com o Golbery?

\falaM Quem tinha contato com ele há algum tempo era o Elio Gaspari, quando
ele trabalhava na \emph{Veja}. O Elio foi editor de política na revista
até 1974. O Elio me disse: ``Você precisa conhecer o Golbery, uma pessoa
muito interessante''. Estamos no início dos anos 1970. Golbery estava
fora do governo, pertencia ao grupo dos Sorbonnianos…

\falaG Mas nenhum deles tinha elos com a Sorbonne.

\falaM O primeiro foi o Castelo Branco, que tinha lido muito Victor Hugo e
isso o qualificava como Sorbonniano. Não conheci pessoalmente o Castelo
Branco, mas receio que fosse um idiota. Mas não importa. As minhas
opiniões valem até um certo ponto. O fato é que conheci o Golbery. Fui
ao Rio e o conheci em 1972, quando já era editor há quatro anos da
\emph{Veja}. Naquele momento ele era presidente da Dall Chemical no
Brasil. O superior dele era um homem chamado Orefici, que os
norte"-americanos chamavam de Orficce. Golbery era um homem cordial, uma
pessoa muito simpática. Acabamos confraternizando. E assim nasceu o
hábito de eu ir ao Rio visitá"-lo, ainda como presidente da Dall
Chemical. Íamos almoçar em uma dessas típicas padarias portuguesas
populares do Rio, onde se comia um bom bacalhau, por exemplo. Ele foi
uma fonte fundamental. E simpatizei com ele. Nossas ideologias eram
opostas. Mas ele gostava, como eu, de arte. Um ponto em comum. Tenho
certeza de que ele simpatizou muito comigo. Falava muito bem de mim.

\falaG À época o Golbery já tinha uma estratégia para devolver o governo aos
civis?

\falaM Estava tramando. Havia começado, pouco a pouco, e pretendia devolver
tudo aos civis. De saída, criou um contato com a casa"-grande. Mas veja:
era um homem em perfeita sintonia com o contexto da Guerra Fria,
portanto com a geopolítica dominada pelo império norte"-americano. Por
isso acabou na Dall Chemical. Mas ao mesmo tempo ele achava que devia"-se
colocar a casa em ordem. Para ele, havia uma grande bagunça por conta do
Fidel Castro. Golbery estava longe de ser um homem de esquerda.
Absolutamente. Há quem diga que tinha um passado de meio incerto no
sentido ideológico.(É isso Mino?) Mas não sei se isso é verdade e se
deve ser levado em consideração. Ele era antes de tudo um filho da
Guerra Fria e da divisão maniqueísta do mundo. Quanto a isso não há a
menor dúvida. Agora, era um estrategista notável. Odiado pelos
militares. Tanto que estava tramando a substituição de Médici, na hora
certa evidentemente, ou seja, quando ele terminasse o mandato dele.
Médici seria substituído pelo Geisel. O Geisel era um sujeito que tinha
aos olhos dos militares a enorme vantagem de ser irmão do Orlando
Geisel.

\falaG Por que vantagem?

\falaM Porque Orlando Geisel era uma espécie de condestável do poder
ditatorial. De garantia. Mas ao mesmo tempo, o Golbery, homem muito
sagaz, percebia que o Geisel era o títere ideal para o titereiro
Golbery. Na verdade, o Golbery nunca me disse que o Geisel era o títere
ideal para ele. Mas depois a certa altura me disse algo muito
significativo a respeito do Geisel. Apesar de todos os livros que o Elio
Gaspari escreveu a respeito do Geisel, porque o ex"-secretário do Geisel,
Heitor Aquino, deu"-lhe todos os arquivos. Então o Elio escreveu, não
sei, quatro livros, monumentais. Neles, Geisel, para o Gaspari, se
tornou o herói. As histórias brasileiras são extraordinárias. Mas
voltemos ao Golbery. Ele preparava então a substituição do Médici que se
deu efetivamente no decorrer de 1973. Uma condição colocada pelo Médici
foi que o Geisel não levaria o Golbery para o governo. Portanto, a
operação foi muito escondida. Estranha de certa forma. E acabou com a
escolha do Geisel pelo Médici. Para abreviar, formou"-se o governo do
Geisel e o Golbery foi naturalmente para a chefia da Casa Civil, em
1974. A relação entre Golbery com o Geisel era bastante formal.
Tratavam"-se de senhor. E o Golbery sabia que conseguiria levar o Geisel
na corrente dele. Qual era o plano do Golbery? Partir para uma anistia
incompleta, mas anistia. Esses seriam os dois passos finais do Geisel:
anistia e reforma partidária. Anistia no começo de 1979 e a reforma
partidária no final de 1979. Ai a coisa marchou.

\falaG Em 1974, você nos levou, a Manuela e eu, a Brasília quando foi se
encontrar com o Golbery.

\falaM Exato. Estava presente fora da sala do Golbery o Roberto Civita. Ele
me implorou para entrar porque ele não conhecia o Golbery. Eu disse:
``Você entra, mas fica quieto''. O Roberto ficou quieto. Mas na hora da
despedida, ele disse ao Golbery: ``Se o senhor quiser a cabeça do Millôr
Fernandes…'' O Golbery o interrompeu e disse: ``Senhor Civita, eu não
estou pedindo a cabeça de ninguém''. E quando saímos da sala, eu disse
ao Roberto: ``Você é um cretino''. E ele ficou quieto. Era um imbecil.

\falaG Lembro da cena. E como continua o plano de Golbery de entregar o
poder aos civis?

\falaM Em 1974 o Quércia se elegeu senador. O MDB colocou deputados e
senadores em todos os cantos. Houve cassações às pamparras. Fui ter com
o Golbery e ele disse: ``Olha eu sou muito parlapatão, mas veja, não tem
jeito. Precisamos fazer uma reforma partidária. Quem é de esquerda que
fica à esquerda, quem é de direita que fique à direita. Que fique tudo
mais claro''. No ano seguinte, 1975, o Golbery teve um descolamento da
retina. Foi operado na Espanha, onde na época havia os bambas desse
problema. Golbery voltou ao Rio de Janeiro. Ficou entrevado em um
quarto, onde passou um mês inteiro na escuridão. Finalmente voltou a
Brasília. No dia 3 de abril, aproveitando"-se da ausência de Golbery,
Geisel havia pronunciado o discurso da pá de cal. Ou seja, estava
encerrada a distensão lenta e gradual, porém segura. Fui visitar o
Golbery no dia seguinte ao pronunciamento de Geisel. Jazia sobre a mesa
do Golbery o discurso do Geisel. Golbery me disse: ``Sabe quem escreveu
isso aqui? O João Paulo dos Reis Veloso''. Reis Veloso era então o
ministro do Planejamento de Geisel. Golbery tinha sublinhado o discurso
inteiro. ``Isso vai provocar o retorno do terror de Estado. Vai ter
coisas do arco da velha. É uma estupidez total. Cuide"-se'', avisou
Golbery. Iam pegar jornalistas. Culminou com a morte do Vlado Herzog, em
outubro de 1975. O Golbery também me disse: ``Olha, se o Geisel não
ficar atento ele não vai fazer o sucessor dele. E o sucessor dele será o
Frota. E ai?, indaguei. ``O Frota tem que cair.'' Quando? Já? Amanha?
``Até o dia 12 de outubro de 1977'', respondeu o Golbery. Estávamos em
agosto de 1975. O Frota caiu exatamente no dia 12 de outubro de 1977.
Dois anos depois, o Figueiredo fez a anistia, bastante parcial. A
anistia foi aprovada pela chamada Comissão da Verdade, pelo Congresso
brasileiro e pelo Supremo Tribunal Federal. E Golbery conduziu a reforma
partidária em 1979. O Tancredo formou imediatamente o Partido Popular. O
Lula formou o Partido dos Trabalhadores. o Brizola, de quem tinham
roubado o Partido Trabalhista Brasileiro, fez o PDT. E o Ulysses ficou
com o PMDB. Depois das bombas do Riocentro, em 1981, o Golbery disse a
Figueiredo: ``É preciso demitir o general Gentil Marcondes''. Marcondes
era então o comandante do Primeiro Exército, e, portanto, o primeiro
mandante das bombas do Riocentro. Figueiredo rebateu: ``Não, não, vamos
chamar o Otávio''. Veio o general Otávio Medeiros, a quem o Golbery uma
vez me apresentou como camarada Dimitrov. Ele tinha muito senso de
humor.

\falaG Quando o Golbery disse para você se cuidar como jornalista após o
pronunciamento de Geisel em abril de 1975 ele sabia que pediriam a tua
cabeça… \falaM Claro. Para os Civita estava em jogo o empréstimo pedido a
Caixa Econômica Federal, presidida pelo Rieschbieter. (É isso mesmo?)
Tratava"-se de um empréstimo de 50 milhões de dólares. De fato, quando
chegou a minha hora por causa desse empréstimo, em 1976, o Rieschbieter
teve de entregar a aprovação a um superior. O pedido de empréstimo
empacou na mesa do Falcão, o ministro da Justiça. O Geisel, diga"-se, me
odiava.

\falaG Ele disse que você era o mais chato de todos os jornalistas.

\falaM Não, esse foi o Figueiredo. Me fez um grande elogio. Ele me confundiu
com o Roberto Marinho e o Victor Civita. Figueiredo disse: ``Esses só
vêm me pedir coisas. O Mino Carta não pede nada. E um chato. Rescreveria
os Evangelhos. Mas ele não tem o rabo preso''. Foi o maior elogio que já
recebi. Mas ele me confundia com donos de empresas, o Roberto Marinho e
o Victor Civita. De qualquer modo, quando, em fevereiro de 1976, o
Geisel disse que não tinha saído o empréstimo de 50 milhões de dólares
para a Editora Abril, ele pediu a minha cabeça e você sabe como. Saí da
\emph{Veja} e acabou a censura contra a revista. Os Civita receberam 50
milhões de dólares. Os Civita deviam sobretudo ao Morgan Trust. Por isso
digo, sou muito mais caro do que Jesus. O Golbery não mexeu uma palha
quando fui embora da \emph{Veja}. Nas suas memorias, o Rieschbieter
conta que foi ter com o Golbery e disse: ``Mas o Mino?'' E o Golbery
respondeu: ``O Geisel detesta ele. Não tem jeito. Não posso fazer nada.
Não posso defender o Mino. Não posso dizer deixem o Mino em paz''. Fui
trabalhar com o Luís, meu irmão, e com o Domingo Alzugaray em uma
revista mensal que se chamou \emph{IstoÉ.} Anódina, inodora e mensal.
Primeira capa foi o Beethoven, sobre os surdos. A \emph{IstoÉ} virou
semanal quando acabou a censura, em 1977. O último a ser censurado foi o
jornal do Dom Paulo Evaristo Arns. O \emph{Jornal da Cúria Metropolitana
de São Paulo}. Dom Paulo era outra pessoa odiada pelo Geisel.

\falaG E com a \emph{IstoÉ} semanal veio tua amizade com o Lula.

\falaM Ficamos logo amigos. Amigos de jantar um na casa do outro.

\falaG Você me levou algumas vezes para comer frango com polenta na casa do
Lula.

\falaM Me lembro. E a Angélica (mulher de Mino) também ficou muito amiga da
Marisa, mulher do Lula. Eu também simpatizava muito com a Marisa. Era
descente italiana com cidadania italiana, pois tinha o passaporte
italiano. E o Lula chamava a avó dela de \emph{nonnina}. A Marisa era
uma mulher muito simpática. E vigorosa. Uma noite, tempo do mensalão, o
Lula me liga e diz: ``Vem jantar aqui na Granja do Torto'' Aí ocorreu
aquela história que já te contei em que a Marisa falava ao Lula que ele
se deixou enganar pelos responsáveis pelo ``mensalão''. O Lula me disse
que deveríamos fazer uma grande entrevista. Fiz uma entrevista de 13
páginas para a \emph{CartaCapital}. Isso em 1995. Eu tinha estado pouco
antes visitando o chefe da Polícia Federal, o Paulo Lacerda. Um delegado
até refinado. Fui com o Sergio Lírio e o Belluzzo. E a certa altura eu
disse ao Lacerda. ``Escuta. E essa história da Operação Chacal (Mino, é
preciso explicar isso). Pegou o disco rígido do Daniel Dantas?'' Ele
respondeu que o disco rígido tinha sido entregue à Ellen Gracie,
ministra do STF. E aí? O delegado respondeu: ``Está vendo aquela gaveta,
o disco rígido está ali dentro''. E o senhor recebeu pressões para
enterrar a operação Chacal? Há, muitas de deputados, senadores.
Ministros? ``Sim''. Por exemplo, o José Dirceu? ``Sim'', respondeu ele.
Já contei essa história. Não respeitei o \emph{off}. Eu perguntei ao
delegado Lacerda: ``Mas desculpe, por que não abrem o disco rígido?''
Ele disse, ``Se abrirem acaba a República''.

\falaG E como foi a entrevista de 13 páginas com o Lula?

\falaM Nessa entrevista, o Lula me disse: ``Você sabe muito bem que não sou
de esquerda''. Eu disse, pensando em Bobbio, na ideia de quem quer
igualdade é de esquerda. E indaguei: ``Não e essa a tua preocupação? E o
Lula respondeu que se isso era ser de esquerda então ele é de esquerda.

\falaG Um homem que faz uma política doméstica voltada para o povo é
naturalmente de esquerda.

\falaM E com a independência do Brasil no cenário internacional. São as duas
questões centrais do governo do Lula. Claramente de esquerda, e com o
ideário de esquerda. Sim, claro, ele sempre foi de esquerda sem ser
envolvido com ideologia marxista e questões similares. O PT nasceu como
um partido de esquerda com ideário de esquerda. Mas repito: o partido no
poder se portou como os outros. Hoje o partido está crescendo novamente.
Na esteira do Lula, é claro. Mas ainda é um partido fraco, tíbio,
amedrontado, sempre na defensiva.

\falaG Já disse ao Lula isso?

\falaM Sempre. Ele me lê, sabe. Ele diz que os trabalhadores brasileiros
saberão reagir. Mas onde? Acho que a única voz que pode criar um
forrobodó do capeta é a dele. Aí sim. Só ele pode engajar o povo. Porque
o discurso dele pega. Ele é um mobilizador. Sem dúvida. Muito eficaz.
Agora, eu tenho por ele uma amizade incondicional, mas o critico
abertamente. No entanto, sempre estive ao lado dele. Quando ele foi
preso, em 1980, fui visitá"-lo com o Raymundo Faoro, com quem eu já tinha
estado no palanque. No DOPS fomos muito bem recebidos pelo então chefe
da PF, o Tuma. Ele nos deixou à vontade na sala do Lula. O Faoro disse
ao Lula, então não somente preso, mas também enquadrado na lei de
Segurança Nacional. E o Faoro disse: ``Olha, estou a disposição para ser
seu advogado''. E o Lula disse: ``Não, doutor Faoro. Isso é coisa pouco
importante. Quem vai me defender é o Greenwald. O senhor não se
preocupe''. Depois ele queria o Faoro na chapa dele como candidato à
vice"-presidência, acho que em 1998.

\falaG Como estava o Lula no DOPS?

\falaM Tranquilo. Mas estava sendo muito bem tratado. O Tuma não somente
mandava chamar a Marisa e os filhos todos os dias. Fui com a Angélica ao
enterro da mãe do Lula. O Tuma soltou o Lula acompanhado por dois
agentes à paisana para ir ao enterro. E depois servia ao Lula lulas
fritas todos os dias. O Tuma gostava muito do Lula.

\falaG O Lula é uma força da natureza. Teve um câncer na garganta. Perdeu a
mulher. E está sendo condenado sem provas.

\falaM Sim, é um homem forte. Lidou com o câncer sem se lamentar. Perdeu a
Marisa para a Lava Jato. E está sendo julgado sem provas. Quando o
encontro, ele me diz que tenho de ficar animado, esperançoso porque a
esperança não morre nunca. Eu digo a ele que ele é claramente sólido.
Estranhamente, um mês antes da morte da Marisa, cheguei para o Lula e
disse: ``Sabe que eu tenho sonhado muito com a Marisa. Me lembro daquele
dia em que estávamos naquele bar embaixo de sua casa quando você morava
atrás da Volkswagen. A Marisa estava presente, e a certa altura chegou a
perua da revista para me pegar. Fui embora e você ficou no bar. Mas a
Marisa saiu do bar e ficou me acenando. Eu sonho com essa cena: eu saio
do carro e ela saudando. É um sonho recorrente''. E ela morreu um mês
depois.

\falaG O Mazzini, como já falamos, dizia que era fundamental repetir as
histórias para as pessoas as entenderes. Nessa nossa longa conversa você
repetiu várias vezes ``é a hora da revolta''. E o Lula é o homem que
deveria engajar o povo.

\falaM O Lula e a única pessoa, o único líder popular nacional capaz de
desencadear uma resistência total a tudo que está acontecendo. Esse
País, sem revolta, afunda nas mãos dessa gentalha. Não vejo outra saída
possível. Sinceramente. Tudo começa, e' claro, com a condenação de Lula
em segunda instância. Isso não pode acontecer. Não pode acontecer. Tem
que haver luta contra isso. O povo na rua disposto a brigar. Caso
contrário acho sinceramente que não haverá eleição. É somente sem
eleição que as quadrilhas continuarão no poder a benefício dos ricos e
super ricos. Eles não terão um candidato em qualquer situação que seja.
Qualquer um será derrotado. Trata"-se de uma situação realmente
paradoxal. O povo está dando 3 por cento de aprovação a esse senhor
Temer. No comando, o está fazendo misérias. Está liquidando o País. E
não há retorno. Uma vez que você vendeu o pré"-sal você vendeu o pré"-sal.
Eles vão devolver o pré"-sal? Se está vendido ao capital estrangeiro
vamos ter de lidar com companhias como a Exxon. Não tem saída sem a
revolta. É hora de reagir.
