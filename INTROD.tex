\chapterspecial{E agora, Luis Inácio?}{}{Paulo Henrique Amorim}




Nenhum jornalista conhece o Lula tão bem quanto o Mino.
Nenhum politico conhece o Mino tão bem quanto o Lula.

O Lula, porém, já teve uma visao utilitária do jornalismo.
“Crítica boa é a que é a favor.”

Como disse o general"-presidente Costa e Silva à Condessa Pereira Carneiro, dona do Jornal do Brasil.
Segundo o que ela disse ao general"-presidente Figueiredo, na minha frente.

Depois, operário"-presidente eleito, Lula não se deu conta do poder destruidor da “midia nativa” — como diz o Mino — , especialmente o da Globo.
Da critica sórdida, invariavelmente contra.
E dela foi uma das vitimas.

Mino, por sua vez, jamais entrará naquele casebre que enchia de agua até os joelhos na temporada de chuva de verao em São Paulo.
Nem, como eu, faz ideia do que significa só ter comido pão aos treze anos de idade.
O Mino e eu não captamos – como diz o Mino- a “secreta essencia” do que é ser pobre.
Mas, Lula e Mino estao e estiveram sempre juntos, do mesmo lado da linha do trem.

Por isso, o Lula, provavelmente, conte ao Mino e a nenhum outro jornalista o que o Mino até já sabe.
O Lula não faz ideia, provavelmente, dos quatro, ou cinco lances subsequentes do xadrez do Mino, quando o Mino critica o Lula para salvar o Brasil.

O Mino não faz ideia, provavelmente, dos quatro ou cinco lances subsequentes do xadrez Lula, quando ele salva o pobre sem tirar do rico.

Apesar das aparencias, Mino confia no povo brasileiro.

Porque ele confia no Lula.
\medskip

Paulo Henrique Amorim, que confia nos três





