\chapterspecial{Prefácio}{}{Paulo Henrique Amorim}



Nenhum jornalista conhece o Lula tão bem quanto o Mino. Nenhum político conhece
o Mino tão bem quanto o Lula. Mas isso não significa que um conheça
inteiramente o outro.

O Lula, por exemplo, já teve uma visão utilitária do jornalismo.

“Crítica boa é a que é a favor.”

Como disse o general-presidente Costa e Silva à condessa Pereira Carneiro, dona
do \emph{Jornal do Brasil}.  Segundo o que ela disse ao general-presidente Figueiredo,
na minha frente.

Depois, operário-presidente eleito, Lula não se deu conta do poder destruidor
da “mídia nativa” – como diz o Mino –, especialmente o da Globo.  Da crítica
sórdida, invariavelmente contra. E dela foi uma das vítimas. Assim como a
\emph{CartaCapital}. Por sua vez, Mino, como eu, jamais entrará naquele casebre que
enchia de água até os joelhos na temporada de chuva de verão em São Paulo. Nem,
como eu, Mino faz ideia do que significa só ter comido pão aos 13 anos de
idade.

Mino e eu não captamos – como diz o Mino – a “secreta essência” do que é ser
pobre. Mas Lula e Mino estão e estiveram sempre juntos, do mesmo lado da linha
do trem.

Por isso, Lula, provavelmente, conte ao Mino e a nenhum outro jornalista o que
Mino até já sabe.

O Lula não faz ideia, provavelmente, dos quatro ou cinco lances subsequentes do
xadrez do Mino, que, às vezes, o critica com a melhor das intenções. Mas Mino
não faz ideia, provavelmente, dos quatro ou cinco lances subsequentes do xadrez
de Lula.

Apesar das aparências, Mino confia no povo brasileiro.

Porque ele confia no Lula.

E o povo confia no Lula.

Desconfio, até, que o povo espere que Lula ouça mais o Mino e entre na
casa-grande com o pé na porta, com força – e pela porta da frente!



