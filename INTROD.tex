\chapterspecial{Prefácio}{}{Paulo Henrique Amorim}



Nenhum jornalista conhece o Lula tão bem quanto o Mino. Nenhum político conhece
o Mino tão bem quanto o Lula. Mas isso não significa que um conheça
inteiramente o outro.

O Lula, por exemplo, já teve uma visão utilitária do jornalismo.

“Crítica boa é a que é a favor.”

Como disse o general-presidente Costa e Silva à condessa Pereira Carneiro, dona
do \emph{Jornal do Brasil}.  Segundo o que ela disse ao general-presidente Figueiredo,
na minha frente.

Depois, operário-presidente eleito, Lula não se deu conta do poder destruidor
da “mídia nativa” – como diz o Mino –, especialmente o da Globo.  Da crítica
sórdida, invariavelmente contra. E dela foi uma das vítimas. Assim como a
\emph{CartaCapital}. Por sua vez, Mino, como eu, jamais entrará naquele casebre que
enchia de água até os joelhos na temporada de chuva de verão em São Paulo. Nem,
como eu, Mino faz ideia do que significa só ter comido pão aos 13 anos de
idade.

Mino e eu não captamos – como diz o Mino – a “secreta essência” do que é ser
pobre. Mas Lula e Mino estão e estiveram sempre juntos, do mesmo lado da linha
do trem.

Por isso, Lula, provavelmente, conte ao Mino e a nenhum outro jornalista o que
Mino até já sabe.

O Lula não faz ideia, provavelmente, dos quatro ou cinco lances subsequentes do
xadrez do Mino, que, às vezes, o critica com a melhor das intenções. Mas Mino
não faz ideia, provavelmente, dos quatro ou cinco lances subsequentes do xadrez
de Lula.

Apesar das aparências, Mino confia no povo brasileiro.

Porque ele confia no Lula.

E o povo confia no Lula.

Desconfio, até, que o povo espere que Lula ouça mais o Mino e entre na
casa-grande com o pé na porta, com força – e pela porta da frente!





\chapterspecial{Apresentação}{}{Mino Carta}

O Brasil nunca viveu tempos iguais aos desencadeados pelo golpe de 2016,
nem mesmo nos 21 anos de ditadura. Explico. Os golpistas de então
armaram uma arapuca para si mesmos, presas da típica hipocrisia nativa,
inventaram um sistema eleitoral e até o \versal{AI}-5 mantiveram o Congresso em
atividade, fecharam"-no para reabri"-lo mais tarde. Houve oposição
valente, e o Ato Institucional foi sua consequência. Daí em diante, as
cassações multiplicaram"-se, nem por isso a resistência parlamentar
arrefeceu. O \versal{MDB} liderado por Ulysses Guimarães ofereceu abrigo a todos
os opositores e nas eleições de 1974 colheu vitórias significativas. Foi
neste momento que o general Golbery começou a cogitar da reforma
partidária concretizada cinco anos depois, com o propósito de estilhaçar
a aliança oposicionista. E nem assim deu certo.

Houve também a resistência armada dos inconformados irredutíveis, e o
resultado foram mais de 400 assassínios e a tortura de milhares. Mas
houve também a sedimentação da esperança de muitos de que algum dia
finalmente raiaria o sol da liberdade. E resistência houve,
extraordinária, com as greves do \versal{ABC}, de 1978, 79 e 80, e o surgimento
da liderança de Luiz Inácio da Silva, dito Lula, à testa de uma nova
leva de sindicalistas habilitados a substituir os pelegos. Vale dizer,
ainda, que os militares souberam impor seu nacionalismo à casa"-grande,
que os convocara para o golpe de 1964. Com isso, ao menos, o País não
sofreu o risco do entreguismo.

A situação atual nasce de uma hipocrisia infinitamente mais descarada do
que aquela que orientou a ditadura, com o desplante de se vestir de
legalidade. Poderia, no entanto, ser de outra forma se o golpe de 2016
foi desfechado pelos próprios poderes da República? Falsos os motivos do
\emph{impeachment}, Constituição rasgada sem a mais pálida interferência
dos guardiões da lei, enquanto o Legislativo empossava o presidente
ilegítimo, herói inconteste da corrupção generalizada.

Objetivo do golpe: tornar Lula inelegível graças à Inquisição do Santo
Ofício de Curitiba e Porto Alegre, pronta a condenar sem provas o
ex"-presidente com a bênção de Tio Sam. Entende"-se: tal é a garantia de
transformar o Brasil em Estado mínimo neoliberal, sujeito às
conveniências de Washington e ao capital estrangeiro. A venda do
pré"-sal, que avalizava o futuro do País, já está selada, de sorte a
preparar a privatização da Petrobras, da qual se tirou o bem maior. Na
pauta, também a privatização da Caixa Econômica Federal, e isso tudo a
preços de liquidação.

A reforma trabalhista, que nos devolve ao passado remoto, entrou em
vigor em novembro para alegria da Fiesp no mesmo instante em que os
ruralistas se regozijam com as chances oferecidas à sua vocação
escravagista. A intolerância reina em todos os aspectos mais
retrógrados, a confirmar a medievalidade verde"-amarela. De fato, a
casa"-grande consegue se afirmar da maneira inatingida com a ditadura.
Tudo se faz para favorecer os ricos e os super"-ricos, em nome de um
internacionalismo que agrada apenas ao capital.

Haveria de ser a hora da indignação e da revolta, mas o povo espezinhado
está inerte e os trabalhadores, resignados.



