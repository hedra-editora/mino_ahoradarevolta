\part{A Hora da Revolta}


\chapterspecial{Apresentação}{}{Mino Carta}

O Brasil nunca viveu tempos iguais aos desencadeados pelo golpe de 2016,
nem mesmo nos 21 anos de ditadura. Explico. Os golpistas de então
armaram uma arapuca para si mesmos, presas da típica hipocrisia nativa,
inventaram um sistema eleitoral e até o \versal{AI}-5 mantiveram o Congresso em
atividade, fecharam"-no para reabri"-lo mais tarde. Houve oposição
valente, e o Ato Institucional foi sua consequência. Daí em diante, as
cassações multiplicaram"-se, nem por isso a resistência parlamentar
arrefeceu. O \versal{MDB} liderado por Ulysses Guimarães ofereceu abrigo a todos
os opositores e nas eleições de 1974 colheu vitórias significativas. Foi
neste momento que o general Golbery começou a cogitar da reforma
partidária concretizada cinco anos depois, com o propósito de estilhaçar
a aliança oposicionista. E nem assim deu certo.

Houve também a resistência armada dos inconformados irredutíveis, e o
resultado foram mais de 400 assassínios e a tortura de milhares. Mas
houve também a sedimentação da esperança de muitos de que algum dia
finalmente raiaria o sol da liberdade. E resistência houve,
extraordinária, com as greves do \versal{ABC}, de 1978, 79 e 80, e o surgimento
da liderança de Luiz Inácio da Silva, dito Lula, à testa de uma nova
leva de sindicalistas habilitados a substituir os pelegos. Vale dizer,
ainda, que os militares souberam impor seu nacionalismo à casa"-grande,
que os convocara para o golpe de 1964. Com isso, ao menos, o País não
sofreu o risco do entreguismo.

A situação atual nasce de uma hipocrisia infinitamente mais descarada do
que aquela que orientou a ditadura, com o desplante de se vestir de
legalidade. Poderia, no entanto, ser de outra forma se o golpe de 2016
foi desfechado pelos próprios poderes da República? Falsos os motivos do
\emph{impeachment}, Constituição rasgada sem a mais pálida interferência
dos guardiões da lei, enquanto o Legislativo empossava o presidente
ilegítimo, herói inconteste da corrupção generalizada.

Objetivo do golpe: tornar Lula inelegível graças à Inquisição do Santo
Ofício de Curitiba e Porto Alegre, pronta a condenar sem provas o
ex"-presidente com a bênção de Tio Sam. Entende"-se: tal é a garantia de
transformar o Brasil em Estado mínimo neoliberal, sujeito às
conveniências de Washington e ao capital estrangeiro. A venda do
pré"-sal, que avalizava o futuro do País, já está selada, de sorte a
preparar a privatização da Petrobras, da qual se tirou o bem maior. Na
pauta, também a privatização da Caixa Econômica Federal, e isso tudo a
preços de liquidação.

A reforma trabalhista, que nos devolve ao passado remoto, entrou em
vigor em novembro para alegria da Fiesp no mesmo instante em que os
ruralistas se regozijam com as chances oferecidas à sua vocação
escravagista. A intolerância reina em todos os aspectos mais
retrógrados, a confirmar a medievalidade verde"-amarela. De fato, a
casa"-grande consegue se afirmar da maneira inatingida com a ditadura.
Tudo se faz para favorecer os ricos e os super"-ricos, em nome de um
internacionalismo que agrada apenas ao capital.

Haveria de ser a hora da indignação e da revolta, mas o povo espezinhado
está inerte e os trabalhadores, resignados.



\chapter*{}

\epigraph{``Pessimismo da ação, otimismo da vontade''}{Antonio Gramsci}

%\newcommand[1]{\fala}{\noindent\textbf{#1}}

\parindent0pt
\parskip\medskipamount

\versal{Gianni:} Então, esta haveria de ser a hora da revolta?

\versal{Mino:} Claro, se este não fosse o Brasil. Nós somos os
revoltados, integramos uma minoria exígua. O povo vive entre a
resignação atávica e o medo da chibata.

%\textbf{Gianni:}
\itshape
 Mas Lula saberia mobilizar o povo como, quando
presidente do Sindicato dos Metalúrgicos de São Bernardo e Diadema,
mobilizou os seus comandados e 100 mil brasileiros lotaram a Vila
Euclydes por longos dias a fio.

%\textbf{Mino:} 
\normalfont
Sim, já soube mobilizar, e até hoje ele sabe falar ao
povo de forma incomparável, haja vista as caravanas no Nordeste, Minas,
Rio de Janeiro e Espírito Santo. Receio, porém, que ele tenha percebido
que uma coisa é mobilizar para o voto, ou para uma festa sem riscos -- e
o povo brasileiro é profundamente festeiro --, e outra para a revolta. É
uma interpretação dos sentimentos dele que me permito fazer, tenho por
Lula uma amizade inoxidável de 40 anos, incondicional, o que não impede
que, em determinadas ocasiões, possa criticá"-lo. Quando ataca a
casa"-grande eu me regozijo. Agora Lula se propõe a enfrentar a injustiça
e a ilegalidade pretensamente legal. E como candidato, alheio à ameaça
da condenação iminente, abre fogo contra os seus inimigos, o mercado e a
Rede Globo, como capitânia da mídia nativa, e promete uma nova ``Carta
aos Brasileiros'', com orientação oposta daquela Carta que, em 2002,
Palocci lhe serviu para agradar à casa"-grande. Trata"-se de um passo
muito importante, a significar a convicção de que a ideia da conciliação
foi sepultada. Lula adianta que na nova Carta exporá o programa do seu
governo, com inclusão de um item crucial, a taxação dos ricos, em busca
da distribuição da riqueza, objetivo mais amplo e profundo da
distribuição da renda. Documento fundamental, um legado para o Brasil do
futuro, ou, quem sabe, centelha que inflama o povo.

%\textbf{Gianni:}
\itshape
 Marcos Coimbra, presidente do Instituto Vox Populi,
afirma que somente Lula hoje poderia devolver o Brasil à rota
democrática.

%\textbf{Mino:} 
\normalfont
Concordo plenamente, conciliação somente entre as
próprias elites. Os inquilinos da casa"-grande podem desentender"-se, e
então a conciliação é necessária entre eles, a bem dos objetivos maiores
da reação. A negociação é impossível. Não há chance de entendimento
entre o Capital e o Trabalho, entre ricos e pobres. Daí a minha
convicção de que Lula será condenado em segunda instância e que as
eleições de 2018 estão ameaçadas: ou as quadrilhas encontram uma solução
``legal'' para realizá"-las, de sorte a criar a enésima exceção, diante
da impossibilidade de emplacar um candidato capaz de dar prosseguimento
ao estado de exceção, ou irão cancelá"-las em nome dos interesses do País
em busca da ``modernização''. Nos dois casos, sofreríamos um golpe
dentro do golpe. Pergunto aos meus desesperançados botões: seria
admissível que as quadrilhas, capazes de fazer o que bem entendem até
hoje, desistissem subitamente de prosseguir na demolição do Brasil,
tomados de um repentino impulso democrático? Os botões nem ousam
responder.

%\textbf{Gianni:}
\itshape
 Você disse que o objetivo principal era tornar Lula
inelegível.

%\textbf{Mino:} 
\normalfont
No meu entendimento, é preciso compreender que retirar
Lula deste páreo sem provas, por meio de uma condenação cominada pelos
Tribunais do Santo Ofício, implica um pleito ilegítimo. Este é o
primeiro ponto a ser encarado. Sem Lula este pleito não vale nada. Esta
é a questão central, a meu ver. Agora, é claro que a situação implica
riscos graves. Por quê? Porque, se você observar o futuro eleitoral,
perceberá que nem o \versal{PMDB} nem o tucanato têm condições de eleger um
candidato capaz de garantir a sobrevida da casa"-grande e da senzala. Por
isso temo um golpe dentro do golpe para impedir as eleições de 2018. E
com o apoio dos Estados Unidos da América.

%\textbf{Gianni:}
\itshape
 Lula, provavelmente, será condenado em segunda
instância. Um magistrado rio"-grandense é padrinho de casamento de Sergio
Moro e com ele fez mestrado.

%\textbf{Mino:} 
\normalfont
Não há saída para o País sem um enfrentamento. Um grande
abalo social. Forte. Um terremoto, se for o caso. Os Pactos de la
Moncloa, que houve na Espanha após a morte de Francisco Franco, em 1975,
estabilizaram o processo de transição para uma democracia. Foram
conduzidos por um homem de direita que, no entanto, admiro bastante,
Adolfo Suárez. Pois bem, os Pactos de la Moncloa, de outubro de 1977,
entendimento entre Capital e Trabalho, no Brasil são totalmente
impossíveis, porque a casa"-grande não aceita acordos com o trabalhador.
Veja como a mídia nativa recebeu o retorno ao trabalho escravo.

%\textbf{Gianni:}
\itshape
 Sim, o trabalho escravo encaixa"-se bem com a dicotomia
casa"-grande e senzala. Foi você quem usou pela primeira vez essa
terminologia, ou, digamos, analogia entre a casa"-grande e as chamadas
elites. E a senzala é a maioria pobre. De volta a uma sociedade
escravocrata?

%\textbf{Mino:} 
\normalfont
Não se trata de um retorno porque a casa"-grande, assim
como a senzala, estão de pé. Além de tudo, este governo de Michel Temer,
com a ajuda do Congresso, e diante de uma Alta Corte de fancaria,
fortalece cada vez mais os ricos e os super"-ricos. Estamos falando,
insisto, de um golpe perpetrado pelos Três Poderes da República. Esta
situação só permite uma saída: um forte abalo social.

%\textbf{Gianni:}
\itshape
 Suponhamos que o povo não reaja e Lula seja inelegível
em outubro de 2018. Lula, não há a menor sombra de dúvida, é um
excelente cabo eleitoral. É muito provável que um candidato apoiado por
ele vença o pleito presidencial.

%\textbf{Mino:} 
\normalfont
O candidato que Lula escolher, o Lula da vez, parte
favorito no meu entendimento. Mas, repito, a vitória de um petista
apoiado por Lula no pleito seria fatal para a casa"-grande e a National
Security Agency. Não é que tenham faltado razões para sair às ruas com
disposição para a briga. Os sindicatos já tiveram e têm razões de sobra.
Evidentemente, convém um exame de consciência. As lideranças não
chegaram ao trabalhador. Não o doutrinaram, não o conscientizaram. É
indispensável mostrar ao trabalhador qual é a situação real.

%\textbf{Gianni:}
\itshape
 Acha que esses líderes e o povo entendem o significado
de um \emph{impeachment}? Em recente entrevista, o embaixador e
ex"-chanceler de Lula, Celso Amorim, lembrou que Richard Nixon não foi
destituído, mas renunciou e o cargo ficou para Gerald Ford. Pois Ford
continuou a trilhar o caminho traçado por Nixon. No Brasil, no entanto,
houve um golpe camuflado de \emph{impeachment}.

%\textbf{Mino:} 
\normalfont
Concordo. Rasgaram a Constituição. Além disso, as
chamadas pedaladas fiscais não seriam um motivo para destituir a
presidenta, já que fazem parte da tradição de quaisquer governos. Além
de tudo, não havia pedaladas. O professor Belluzzo foi ao Congresso no
dia da votação e disse: ``O que houve foram despedaladas''. Os deputados
queriam um pretexto. \emph{CartaCapital} publicou recentemente um artigo
de Luigi Ferrajoli, em que um dos mais importantes juristas da Itália e
do mundo mostra com extrema, insuperável clareza, a ilegalidade de tudo
quanto aconteceu no golpe e nas suas consequências.

%\textbf{Gianni:}
\itshape
 Acha que, se Lula tivesse sido chefe da Casa Civil no
início do segundo mandato de Dilma não teria ocorrido o
\emph{impeachment}?

%\textbf{Mino:} 
\normalfont
Certamente, não teria o \emph{impeachment}. Lula é das
massas, o único líder popular nacional verdadeiro. Há lideranças locais
fortes, mas não nacionais. Lula é líder nacional único, dono de um
discurso que comove o povo, ele tem também talento de negociador e
saberia como assegurar a permanência do governo Dilma até o fim. Eu não
tenho a mais pálida sombra de dúvida a respeito.

%\textbf{Gianni:}
\itshape
 O mensalão foi o primeiro golpe contra Lula e o \versal{PT}?

%\textbf{Mino:} 
\normalfont
Uma tentativa de golpe, eu diria. Apesar do mensalão,
Lula se reelegeu folgadamente. E não somente. Deixou seu governo com
87\% de aprovação. Em seguida vem a Lava Jato, uma operação atirada, a
partir da corrupção da Petrobras, contra Lula e o \versal{PT}. É o começo da
montagem do golpe. A manobra se concretiza com a eleição de 2014, em que
Dilma ganha apertadamente.

%\textbf{Gianni:}
\itshape
 O ``mensalão'' é história encerrada. Entretanto, a
direita continua a associar o mensalão ao \versal{PT}, como se fosse algo
inédito.

%\textbf{Mino:} 
\normalfont
Não são de direita as quadrilhas no poder, mídia nativa
incluída. São a reação na sua acepção mais pura. Churchill era de
direita. O ``mensalão'' é o caixa 2. Sempre houve na política
brasileira. Mensalão não é novidade, mas virou novidade porque envolvia
o \versal{PT}. Tempo de mensalão, outubro de 2005, fui ter com Lula na Granja do
Torto. Quando cheguei, o casal Lula da Silva jogava rouba"-montes. Dona
Marisa dizia: ``Ele (\emph{Lula}) acredita nas pessoas''. E Lula
admitia: ``Fizeram besteira''. O problema são as regras para o
financiamento dos partidos. O mensalão, assim como o escândalo da
Petrobras, deu"-se com outros governos anteriores aos de Lula. Não
podemos esquecer da ``privataria tucana''. Vale lembrar também que
Fernando Henrique Cardoso tinha um amigo decisivo, Sérgio Motta. Era um
operador, como dizem no Brasil, formidável. O dinheiro que circulou à
sombra do príncipe dos sociólogos só Deus sabe.

%\textbf{Gianni:}
\itshape
 Como levar a Lava Jato a sério se o Judiciário está
politizado?

%\textbf{Mino:} 
\normalfont
O Judiciário está a serviço da casa"-grande a ponto de
participar ativamente do golpe. E Sergio Moro e a turma dele compõem um
tribunal do Santo Ofício, visto que toda a argumentação levada adiante
para condenar me lembra muito o tribunal que condenou Joana D'Arc. É
igual, por trás havia também razões políticas. Como Bernard Shaw explica
brilhantemente em \emph{Saint Joan}, uma de suas obras"-primas, houve um
acordo entre uma parte da Igreja francesa e os ingleses, até então
ocupando uma porção da França. Era contra os ingleses que Joana
combatia. Selado o acordo, Joana virou herege. Foi queimada viva em
1431. Esse julgamento me lembra muito aquele de Curitiba, quando
convicções substituem as provas. Estou me referindo à condenação de Lula
a nove anos e meio. Nove anos e meio porque o juiz Moro, maligno, quis
lembrar que Lula tem nove dedos e meio.

%\textbf{Gianni:}
\itshape
 Como obteve essa informação?

%\textbf{Mino:} 
\normalfont
É uma convicção minha. É um sentimento que se abriga no
fundo do meu coração e da minha mente. E me espanta entre o fígado e a
alma. Por essas e outras prepotências e arrogâncias deveria haver uma
mobilização dos trabalhadores, de todos os eleitores de Lula. Mas não
há.

%\textbf{Gianni:}
\itshape
 As primeiras perguntas dirigidas às pessoas que se
prestam a fazer as delações premiadas dizem respeito a Lula e ao \versal{PT}.

%\textbf{Mino:} 
\normalfont
Querem a pele de Lula. Esse Santo Ofício de Curitiba
prestou"-se ao jogo sujo de uma forma vergonhosa. Mas agora os homens do
poder já não querem a Lava Jato. Gilmar Mendes, o ministro do Supremo,
não quer. O Temer não quer. Os parlamentares não querem. Claro, estão
todos incriminados. O Temer, aliás, já está condenado. Vai ser preso ao
deixar o mandato. Não sei… me arrisco a duvidar.

%\textbf{Gianni:}
\itshape
 Você diz que um \emph{mea"-culpa} por parte das
esquerdas brasileiras é necessário.

%\textbf{Mino:} 
\normalfont
Quem se disse de esquerda no Brasil nunca chegou
realmente ao povo brasileiro. Veja o caso do ex"-trotskista Antonio
Palocci, que militava na famosa Liberdade e Luta, a Libelu, movimento
estudantil dos anos 1970. Depois, como ministro da Fazenda no governo
Lula, revelou"-se um neoliberal.

%\textbf{Gianni:}
\itshape
 Você fez a primeira capa de Lula, em fevereiro de 1978,
na semanal \emph{IstoÉ}. Naqueles tempos houve greves bem organizadas e
eficazes.

%\textbf{Mino:} 
\normalfont
O \versal{PT} nasceu ali. Aqueles sindicalistas sabiam convocar
greves. Há pouco tempo fiz uma palestra na \versal{CUT} (Central Única dos
Trabalhadores) de São Paulo, onde, aliás, tenho grandes amigos, e
falou"-se na greve geral na capital. Um sindicalista da Baixada Santista
disse: ``Olha, eu não sei mesmo se houve uma greve geral porque até
agora me pergunto se os trabalhadores não saíram de casa por decisão
própria ou porque os ônibus não saíram das garagens''. Essa observação
me tocou profundamente. Quer dizer, foi uma greve geral, sim: parou
praticamente tudo, menos os aeroportos. Mas realmente fiquei em dúvida
sobre a sinceridade e a força real daquele movimento. Repito, não
escasseiam razões para fazer greves gerais, e muito mais que isso, para
realmente mostrar o protesto, exibir a queixa, trombetear a
insatisfação. No entanto, nada acontece. As pessoas parecem levar uma
vida normal. Portam"-se como se nada tivesse acontecido.

%\textbf{Gianni:}
\itshape
 Em recente entrevista com Lula, ele me disse que há
vários ``Lulinhas'' para substituí"-lo.

%\textbf{Mino:} 
\normalfont
Se ficar claro na eleição que o favorito é um petista
indicado por Lula, e não duvido que isso possa acontecer, ocorrerá,
repito, um golpe dentro do golpe. É isso que temos de entender. E é por
isso que insisto: sem o povo na rua, e é muito povo, todos aqueles que
votam em Lula, não há nada. A grande desgraça do Brasil são três séculos
e meio de escravidão, que ainda não terminou para garantir a permanência
da Idade Média sob o manto norte"-americano. Acrescente"-se, para entender
o Brasil, a ausência de uma guerra de independência. Ao contrário de
outros países sul"-americanos, o Brasil nunca teve uma verdadeira
revolução. Sangue na calçada correu muito pouco no Brasil.

%\textbf{Gianni:}
\itshape
 Esqueçamos o D. Pedro I e o suposto grito do Ipiranga,
que até hoje não se sabe se de fato houve. No entanto, entre 1964 e
1985, durante a ditadura, 450 brasileiros foram assassinados pelo terror
de Estado e torturados milhares. E há uma esquerda no Brasil. Existem o
Movimento dos Trabalhadores Sem Terra (\versal{MST}), o Movimento dos
Trabalhadores Sem Teto (\versal{MTST})…

%\textbf{Mino:} 
\normalfont
De fato, os Sem-Teto e o \versal{MST} são movimentos reais e
souberam doutrinar seus militantes. São dois movimentos de esquerda
muito bem liderados por Guilherme Boulos e João Pedro Stedile,
respectivamente, que alcançam seus objetivos. Houve em São Paulo
recentemente uma marcha dos sem"-teto que representa um evento
emocionante e exemplar. Quanto aos 450 assassinados e desaparecidos,
quero lembrar que haveria mais se isso tivesse sido necessário, aos
olhos da ditadura. No Uruguai, à época país de pouco mais de 3 milhões
de habitantes, morreram 5 mil. Na Argentina foram 30 mil. No Chile nem
sabemos quantos. Torturadores brasileiros eram mestres em outros países
assolados por ditaduras. Somos bons em tortura. Muito bons. O nosso
herói é o Duque de Caxias, que comandou o genocídio do Paraguai, no fim
do século \versal{XIX}. E é celebrado em praça pública em estátuas equestres. Não
tivemos Bolívar, nem San Martín e O'Higgins. E por isso somos medievais
até hoje.

%\textbf{Gianni:}
\itshape
 O \versal{PT} foi formado em 1980 por um corajoso movimento de
esquerda.

%\textbf{Mino:} 
\normalfont
Depois das Diretas Já comandadas por Ulysses Guimarães,
aquelas greves são, a meu ver, o movimento de resistência mais
importante à ditadura. Admiro muito a coragem de quem foi para o
Araguaia no fim da década de 60, achando que aquilo era a Sierra
Maestra, e 80 homens enfrentaram 10 mil soldados. Tiro o chapéu, mas
enxergo o lado patético. As greves do \versal{ABC} não foram patéticas, os
trabalhadores foram para a Vila Euclydes para enfrentar bombas e
brucutus. E ao cabo Lula foi preso e enquadrado na Lei de Segurança
Nacional. Sem esquecer a \versal{OAB} quando da presidência de Raymundo Faoro,
meu amigo fraterno, autor de dois livros fundamentais, \emph{Os Donos}
\emph{do Poder} e \emph{A Pirâmide e o Trapézio}. Com Faoro, uma \versal{OAB}
finalmente corajosa e fiel à sua razão de ser, passou a denunciar os
crimes contra os Direitos Humanos cometidos pela ditadura.

%\textbf{Gianni:}
\itshape
 E logo Lula seria preso.

%\textbf{Mino:} 
\normalfont
Lembro da Vila Euclydes cheia até as bordas. Foram
momentos extraordinários. Era o povo na rua. O trabalhador defendia seus
direitos.

%\textbf{Gianni:}
\itshape
 O \versal{PT} tem quase 20\% do eleitorado. Lula concorda ser
maior que o \versal{PT}. Da mesma forma, diz o ex"-presidente brasileiro, Jacques
Chirac era maior que sua sigla de centro"-direita. Willy Brandt foi maior
que a social"-democracia. Em contrapartida, Lula diz que François
Hollande foi menor que o Partido Socialista No entanto, Lula ressalva
que chegou à Presidência sobre os ombros do \versal{PT}.

%\textbf{Mino:} 
\normalfont
Lula tem razão, mas o \versal{PT} no poder, confesso, foi uma
grande decepção. Por quê? Acreditou em conciliação com os neoliberais.

%\textbf{Gianni:}
\itshape
 Creio que Lula tentou ser pragmático. Quis mostrar às
elites que não era um homem perigoso.

%\textbf{Mino:} 
\normalfont
Isso foi perda de tempo. Sabe por quê? Porque os ricos e
os super"-ricos são uma minoria. O povo brasileiro é a maioria. Os pobres
são a maioria. Em lugar de morar debaixo das pontes, eles devem partir
para as calçadas.

%\textbf{Gianni:}
\itshape
 Houve também a espinhosa aliança do \versal{PT} com o \versal{PMDB}.

%\textbf{Mino:} 
\normalfont
Mas aí há uma injunção política. É preciso entender que,
se você quiser governar e contar com uma maioria no Congresso,
evidentemente você terá de ter um aliado. E, no caso, a sigla sempre
disponível é o \versal{PMDB}, o partido do poder. A sigla que se move conforme as
circunstâncias. É normal haver, portanto, um entendimento com o \versal{PMDB}. O
ponto, a meu ver, é outro. Houve demasiadas concessões. E agora Lula
deveria parar de falar bem de Henrique Meirelles na sua pré"-campanha.
Muitas orientações da política econômica dos governos Lula e depois dos
governos Dilma foram neoliberais. Lula colocou Meirelles, porta"-voz do
neoliberalismo nativo, no Banco Central.

%\textbf{Gianni:}
\itshape
 Por vezes, Lula se cercou de ministros preciosos, como
o chanceler Celso Amorim. Ao mesmo tempo, nomeou gente como Meirelles,
Mantega e Palocci. Como explicar isso?

%\textbf{Mino:} 
\normalfont
Lula foi às vezes ingênuo. O caso de Palocci, o
ex"-trotskista da Libelu que virou neoliberal, é um exemplo. Palocci era
o amado prefeito de Ribeirão Preto, onde já havia praticado várias
porcarias. Por exemplo, privatizou até a água da cidade. A política dele
foi neoliberal. Evidentemente, Lula a aprovou. Lula preocupava"-se com a
política social e, por tabela, veio o Bolsa Família, a abertura do
crédito, Minha Casa Minha Vida, o Luz para Todos e outras iniciativas
desse tipo. No entanto, a política econômica do País foi marcada por uma
tendência neoliberal.

%\textbf{Gianni:}
\itshape
 Mas, ao contrário de seus antecessores, Lula
implementou programas sociais importantes para tirar milhões da miséria.

%\textbf{Mino:} 
\normalfont
Como eu disse, houve avanços sociais importantes. Mas nem
por isso deixou de existir nos seus governos uma orientação básica
neoliberal. Isso me parece negativo em uma análise fria. E está na
origem da atual inércia do \versal{PT}. Não houve por parte do partido um
trabalho profundo e capilar no sentido de conscientizar o povo
brasileiro. Foi a ocasião em que isso deveria ter sido feito. Não acho
que a conscientização do povo se resolva simplesmente com o Bolsa
Família e abrindo o crédito para os mais pobres. Foram programas sociais
importantes, claro. Mas qual foi a conquista em termos de consciência da
cidadania de quem estava melhorando de vida?

%\textbf{Gianni:}
\itshape
 A conscientização do povo e os programas sociais
deveriam ter sido um processo concomitante.

%\textbf{Mino:} 
\normalfont
Não há dúvida. Veja o resultado. Sim, houve brasileiros
que entraram no ciclo do consumo. Fala"-se em 35 milhões, 40 milhões. Mas
os ricos e os super"-ricos ficaram mais ricos e super"-ricos. Permaneceu a
distância abissal que marca profundamente a monstruosa desigualdade
neste país. E agora os pobres já voltaram às condições de antes.

%\textbf{Gianni:}
\itshape
 Mas isso por conta das reformas feitas pelo governo
ilegítimo nos últimos 16 meses.

%\textbf{Mino:} 
\normalfont
Estamos diante de um desastre. As reformas atingem os
trabalhadores. Existe uma clara entrega do Brasil ao capital
estrangeiro. Enquanto um imbecil chamado Vargas Llosa, tão endeusado
como escritor, diz que o nacionalismo é o mal dos povos. O que ele quer,
ser Jesus Cristo? Cristo morreu na cruz porque fundou uma religião
universal da igualdade. O nacionalismo, para países como o Brasil, é o
recurso contra a desigualdade neoliberal.

%\textbf{Gianni:}
\itshape
 Lula disse, em outubro, durante a pré"-campanha dele em
Minas Gerais, que, se eleito, submeterá a um referendo as reformas de
Temer.

%\textbf{Mino:} 
\normalfont
Lula seria um presidente que inaugura novamente uma
estação democrática. Os dois mandatos dele como presidente foram
democráticos. Assim como foram os da presidenta Dilma Rousseff, até ela
ser impedida. Nos governos petistas, quero sublinhar, nem tudo a meu ver
foi acerto, porque implementaram políticas econômicas neoliberais. Mas
de qualquer maneira houve avanços sociais e, ainda mais importante, a
meu ver, uma política exterior excelente, de grande independência do
Brasil. E sem forçar a barra inutilmente. Uma política exterior sutil e
muito inteligente.

%\textbf{Gianni:}
\itshape
 O balanço dos governos do \versal{PT} é positivo, o que explica
o nível de aprovação de Lula quando deixou o governo, 87\%. O \versal{PT} baixou
o nível de desemprego para 4,3\%. Durante 12 anos do \versal{PT} no governo,
houve um aumento real de salários para os trabalhadores acima da
inflação. Foram gerados 22 milhões de empregos. Enquanto isso, a Europa
gerou 100 milhões de desempregados. Aqui, 35 milhões saíram da miséria.
De que forma a conciliação de Lula com os neoliberais afetou esses
dados?

%\textbf{Mino:} 
\normalfont
Nada de realmente profundo foi feito. Taxar os ricos, por
exemplo. Nem passou pela cabeça de qualquer governo brasileiro. Mas
agora Lula se dispõe a taxar. Contudo…

%\textbf{Gianni:}
\itshape
 No entanto, Lula não tinha maioria no Congresso para
fazer certas reformas. Foi o caso da regulamentação da mídia. Na França,
por exemplo, o leque de jornais de todas as ideologias existe graças às
subvenções do governo. Lula me disse que fez um congresso para
regulamentar a mídia. Só não compareceram a Globo e a Record. E como ele
não tinha maioria em relação ao controle da mídia e estava no fim do
mandato, passou o caso para o primeiro governo de Dilma.

%\textbf{Mino:} 
\normalfont
Mas, se o País é democrático, você terá inevitavelmente
uma mídia a representar diferentes tendências políticas. O Brasil é o
país da casa"-grande. Aqui há um monopólio disfarçado em oligopólio,
sinal de que existem a casa"-grande e a senzala. Somos totalmente
medievais. Como você regulamenta? De resto, bastaria aplicar a
Constituição de 1988 e executar as leis contra os monopólios.

%\textbf{Gianni:}
\itshape
 Uma empresa não pode ter um número xis de plataformas
midiáticas como jornais, estações de rádio, canais de televisão etc.

%\textbf{Mino:} 
\normalfont
Quando Lula foi para o primeiro mandato, o então
governador do Paraná, Roberto Requião, disse a José Dirceu, chefe da
Casa Civil: ``O governo precisa de uma grande televisão que o apoie''. E
Dirceu: ``Já temos, é a Globo''. Os petistas gostavam era da Globo.
Adoravam aparecer na Globo. A revista \emph{Exame} tinha mais anúncios
do que \emph{CartaCapital}. A \emph{Exame} é quinzenal e dirigida ao
baixo clero das empresas. \emph{CartaCapital} e uma semanal de política,
economia e cultura dirigida para um público geral. A \emph{Exame} tinha
mais anúncios! É para rir.

%\textbf{Gianni:}
\itshape
 No Brasil, você lê um jornal ou assiste a um noticiário
televisivo e é como se tivesse lido todos os jornais e visto todos os
noticiários de tevê. Enquanto isso, as pessoas continuam levando vida
normalmente, como se não vivessem em um estado de exceção. Na França, na
Espanha, na Itália, já estariam nas ruas por muito menos do que ter um
presidente ilegítimo como Temer e viver em um estado de exceções, como
diz Pedro Serrano. Uma vasta maioria dos brasileiros parece não se dar
conta de que não vive em uma democracia.

%\textbf{Mino:} 
\normalfont
Um país com tamanha desigualdade não pode ser
democrático. Temos de entender isso de uma vez por todas. E ter uma
imprensa alinhada toda de um só lado. Que democracia é esta? Onde existe
isso? Vivemos na Idade Média até hoje. A casa"-grande está aí. E a
senzala é visível a olho nu. Vivemos em um regime de exceção
pretensamente vestido de legalidade. A Argentina, ao contrário do
Brasil, é uma democracia. O motivo? Tem uma sociedade muito mais
equilibrada. Comportou"-se desde o século \versal{XIX} de outra forma, com a
presença, entre outros, de San Martín, como já dissemos. Houve uma
verdadeira guerra de independência lá. No século \versal{XX}, houve uma enorme
reação nas ruas contra as ditaduras. Houve sangue nas calçadas. No
Brasil, ditadores são nomes de viadutos. De pontes. É um negócio
inacreditável a nossa leniência. Aqui ainda sofremos as prepotências da
casa"-grande e vivemos nessa desigualdade monstruosa. A desigualdade no
Brasil gera violência. No ano passado, foram mortas mais de 60 mil
pessoas. Sete assassínios por hora. Este país pode ser democrático? E os
ricos e os super"-ricos levantam muralhas em torno de suas vivendas.
Moram em condomínios fechados. Andam de helicóptero com medo de serem
sequestrados e para evitar o trânsito. Seus carros são blindados. Assim
como proliferam os \emph{valet parking} no Brasil, existe uma caterva de
seguranças engravatados, enquanto os patrões deles andam de bermudas. É
um país ridículo. Somos ridículos. Norberto Bobbio define admiravelmente
essa questão. Ou seja, quem é a favor da igualdade é de esquerda. Mas
você tem de ser sincero nessa sua escolha. Não pode ser um palavrório
inútil. Tem de ser algo muito bem definido. No Brasil, quem é a favor da
igualdade com absoluta clareza? O movimento dos sem"-teto, o \versal{MST}, algumas
pessoas certamente. A esquerda brasileira não existe. O \versal{PT} nasceu como
um partido de esquerda, mas no poder portou"-se como os demais

%\textbf{Gianni:}
\itshape
 Nos \emph{Cadernos do Cárcere}, Gramsci fala no impacto
no povo por parte de acadêmicos, intelectuais e jornalistas via
faculdades, textos, cinema, rádio e da imprensa. Ironia das ironias, no
Brasil, essa revolução passiva para atingir ideais hegemônicos
igualitários, no entanto, parece ter sido realizada pelos neoliberais. O
Brasil tem intelectuais capazes de inverter o quadro?

%\textbf{Mino:} 
\normalfont
Há vários: Alfredo Bosi, por exemplo. O Fábio Konder
Comparato. O almirante Othon, preso pela Lava Jato, figura de grande
peso internacional. O Fiori é um observador agudo, grande analista da
política internacional. Assim como Wanderley Guilherme dos Santos e
Roberto Amaral. E Celso Amorim, então? E o Serrano? É o mais arguto
definidor desse golpe e as suas consequências. E o Walfrido Warde. E
Roberto Requião, o formidável senador rebelde? E Lindbergh Farias,
petista atípico? E Marcio Pochmann, Emir Sader. Para não falar dos
jornalistas dignos, como Paulo Henrique Amorim, Luiz Carlos Azenha,
Fernando Morais e Luis Nassif. Temos também Luiz Gonzaga Belluzzo, autor
de uma frase deslumbrante: ``O Brasil é o único país do mundo onde a
luta contra a corrupção leva ao poder os bandidos''.

%\textbf{Gianni:}
\itshape
 No entanto, eles sozinhos não podem fazer esse trabalho
de conscientização.

%\textbf{Mino:} 
\normalfont
Fazem o que podem, com a feroz oposição da mídia.
Conscientizar é um verbo de outros tempos. Queremos que os leitores
falem com seus parentes, seus amigos etc., e digam: ``Olha, gente, vamos
botar a cabeça no lugar''. Mas, no caso específico dos
conscientizadores, acho que a falha, em grande parte, foi do \versal{PT}. E, de
modo geral, as falhas, temos de reconhecer, são de todos aqueles que se
disseram de esquerda e não tiveram atuação política em linha com a
ideologia que diziam professar.

%\textbf{Gianni:}
\itshape
 Lula falou em nossa entrevista que há um declínio da
política. A meu ver, isso ocorre devido à ditadura do capital ter
afastado a política, políticos e partidos do povo. Isso explicaria
movimentos como o ``Indignai"-vos'', de Stéphane Hessel. Teve forte
impacto nos espanhóis que votaram no Podemos. Houve o movimento contra o
mundo financeiro, Occupy Wall Street. Outra agremiação tentou romper com
partidos em sua maioria caducos: o Syriza, uma aliança de movimentos de
esquerda e ecológicos liderada por Tsipras, na Grécia. Talvez, só um
forte movimento nas ruas poderia impedir a condenação de Lula. Repito
uma pergunta anterior, visto que repetir é crucial para que as pessoas
entendam, como dizia o pensador do século \versal{XIX} Giuseppe Mazzini. Esse
gênero de manifestação seria liderado pelo \versal{PT}, sindicatos, \versal{MST}, Sem Teto
e outras siglas progressistas?

%\textbf{Mino:} 
\normalfont
Creio que, antes da sentença pela segunda instância,
seria importante que houvesse manifestações. Dia 12 de novembro, quando
entrou em vigor a nova lei trabalhista com todas as suas maldades e
malignidades, teria sido uma grande ocasião para os trabalhadores saírem
às ruas e fazerem um barulho infernal. Registro alguns episódios
interessantes, diria até divertidos. Por exemplo, o movimento
``tomataço''. Atiraram tomates no ministro Gilmar Mendes, que veio
participar de um evento e assistir a um jogo de futebol aqui em São
Paulo. Mendes aprovou, entre outras coisas, essa concessão notável à
nossa Idade Média, que é o trabalho escravo. Era, na verdade, o que
interessava em primeiro lugar ao Temer, que comprou os deputados da
bancada ruralista para que votassem na permanência dele no poder na
segunda votação no Congresso.

%\textbf{Gianni:}
\itshape
 Gilmar Mendes está te processando.

%\textbf{Mino:} 
\normalfont
Gilmar Mendes me processa na primeira instância. Se por
acaso eu for condenado, recorrerei. Ele me processa por causa de um
editorial. Ali eu tecia considerações que não são caluniosas. E como
quando digo que Gilmar é o Darth Vader brasileiro. Parece"-me um retrato
fiel. Quando põe aquela capa preta, ele encarna o figurino à perfeição.
Eu disse isso à juíza que me interrogava. Ela começou a rir. Há
magistrados bem"-humorados no Brasil.

%\textbf{Gianni:}
\itshape
 Lula diz que, quando assumiu o poder, em 2002, a Bolsa
de Valores de São Paulo tinha 11 mil pontos. Quando ele deixou o
governo, a Bovespa operava com 71 mil pontos. Ele não sabe se agora as
elites estão com raiva dele ``por razões ideológicas ou se é uma questão
de pele''. Por que esse ódio em relação a Lula por parte dessa elite
empresarial e financeira?

%\textbf{Mino:} 
\normalfont
O ódio em relação a Lula existe, em primeiro lugar,
devido a uma pressão norte"-americana, que é puramente ideológica.
Por quê? Porque Lula é
considerado um líder perigoso que representa um certo tipo de esquerda
que não agrada aos Estados Unidos. Esse é o ponto inicial. Mas há também
uma questão de pele. É ódio de classe. Lula representa quem? A senzala.
Aos olhos dessa gentalha, Lula é a senzala. É a figura central da
senzala. Figura que tem esse poder de ganhar eleições como candidato
imbatível. E tem o poder de convocar as massas. Ele mobiliza. Isso é
apavorante. Por essas e outras, eliminemos Lula para o pessoal da
casa"-grande ficar contente. A começar pelos Estados Unidos. O que quer a
casa"-grande? Que o Brasil seja um país de súditos. Querem entregar o
País oferecendo"-o a preço de banana ao capital estrangeiro. Um Brasil
entreguista, súdito de Tio Sam.

%\textbf{Gianni:}
\itshape
 Você falou em ódio de classe.

%\textbf{Mino:} 
\normalfont
No Brasil, o preconceito de classe é brutal. Temos
Sowetos espalhados por todo o País. Quem acaba na cadeia, no Brasil,
são, em primeiro lugar, os negros. Depois os pobres. Coloque, por
exemplo, o ex"-jogador Ronaldo Fenômeno, tão amado, inclusive pelos
ricos, que o veem como um negro de alma branca, à meia"-noite na esquina
do Colégio Dante Alighieri com a Rua Peixoto Gomide, em São Paulo. Passa
a Rota da Polícia Militar e eles o pegam e o atiram dentro da viatura.
Não tem erro.

%\textbf{Gianni:}
\itshape
 Esse racismo envolve a classe social e a cor do outro.

%\textbf{Mino:} 
\normalfont
Aqui existem duas formas de racismo, gravíssimas e
claríssimas. Uma é o racismo racial. O outro é o racismo social. Estão
entrelaçados. Mas, digamos, o pobre branco também está estrepado, devido
ao racismo social.

%\textbf{Gianni:}
\itshape
 Esse sadismo das elites, escreve Gilberto Freyre,
remonta à crueldade dos filhos dos donos da casa"-grande que brincavam
com inaudita violência com os filhos dos negros da senzala. Os meninos
brancos usavam os moleques como cavalos, davam chicotadas neles e por aí
vai. Freyre escreve: ``Não há brasileiro de classe mais elevada, mesmo
nascido e criado depois de oficialmente abolida a escravidão, que não se
sinta aparentado do menino Brás Cubas na malvadeza e no gosto de judiar
do negro''. Freyre fala em ``deleite mórbido''. Isso explicaria esse
atual ódio pelo povo.

%\textbf{Mino:} 
\normalfont
Acho perfeito. A esquerda brasileira execrou Gilberto
Freyre porque era um liberal conservador à moda antiga nas suas posturas
ideológicas. E veja que descrição perfeita. E as esquerdas são contra as
duas obras"-primas de Gilberto Freyre.

%\textbf{Gianni:}
\itshape
 Você prefere \emph{Sobrados e Mocambos}.

%\textbf{Mino:} 
\normalfont
Acho ótimos os dois livros. Nós também temos os sobrados
e mocambos, os palácios e as favelas.

%\textbf{Gianni:}
\itshape
 Segundo Lula, a chave é o mercado interno para que o
povo possa consumir.

%\textbf{Mino:} 
\normalfont
Lula tem razão em acreditar no mercado interno. Esse povo
espezinhado é um tesouro. Se você permite que o povo evolua, você
eliminará o abismo entre pobres e ricos e terá um bom consumidor. Isso
será a força do País. Agora, uma política agrícola é muito menos
importante do que uma política industrial. De qualquer maneira. Sem
contar que o Brasil exporta soja e minério de ferro. E os preços desses
tipos de \emph{commodities} caíram. E agora, como exportador de
petróleo, o Brasil vai deixar de ganhar o que poderia. Privatizada, a
Petrobras irá para a Exxon e outras empresas estrangeiras. O que
interessa é uma política industrial. Mas a indústria brasileira foi
destruída. Este país já foi a décima quinta potência industrial do mundo
a certa altura de sua história, durante uns 30 anos, de Getúlio Vargas a
João Goulart, passando por Juscelino Kubitschek.

%\textbf{Gianni:}
\itshape
 Getúlio criou uma infraestrutura para o desenvolvimento
industrial. Isso é fundamental para a criação de uma economia forte,
como bem sabem os líderes europeus.

%\textbf{Mino:} 
\normalfont
A industrialização que Vargas buscou era na época o
caminho para o desenvolvimento. Dentro desse conceito ele criou, em
primeiro lugar, a \versal{CSN}, em Volta Redonda. Em seguida, criou o salário
mínimo, que à época era superior ao mínimo atual. Por fim, em 1943,
assinou a \versal{CLT}. Em suma, ele criou as bases efetivas para a
industrialização. Essas bases serviram realmente para que o Brasil se
tornasse uma potência industrial. De longe a maior da América Latina.
Foi um período importante. Getúlio depois caiu, mas voltou. E procedeu
no caminho de 1951 a 1954. Criou a Petrobras em 1952.

%\textbf{Gianni:}
\itshape
 Lembro de uma entrevista que fiz com o empresário
Abilio Diniz, em 2012, na qual ele dizia maravilhas sobre Lula e Dilma.
E agora leio que Diniz, Arminio Fraga, Nizan Guanaes e Luciano Huck
criaram um tal ``Foro Cívico'', para apoiar outros candidatos.

%\textbf{Mino:} 
\normalfont
Apoiar os candidatos confiáveis a eles.

%\textbf{Gianni:}
\itshape
 Descartar Lula por um governo ilegítimo e apoiar um
novo candidato neoliberal é uma inenarrável forma de oportunismo. Diniz
badalou Lula no poder e agora se bandeia para o outro lado?

%\textbf{Mino:} 
\normalfont
Trata"-se de gente que não sabe reconhecer os méritos de
Lula. O Arminio Fraga nem coloco nessa equação, pois sempre foi um
neoliberal. No entanto, quando ainda estava no Grupo Pão de Açúcar,
Abilio ganhou dinheiro como nunca na época de Lula, graças à abertura do
crédito.

%\textbf{Gianni:}
\itshape
 Temos também Paulo Skaf, da Fiesp, que exalta a
estabilidade da economia sob Temer.

%\textbf{Mino:} 
\normalfont
Skaf nem é empresário. Trata"-se de uma figura caricata.
De resto, o vídeo e as páginas impressas povoam"-se de rostos
patibulares. Veja as sessões no Congresso ou no Supremo Tribunal
Federal. As reuniões dos supostos representantes do povo e dos guardiões
da lei. São impressionantes a pobreza intelectual, cultural, e o
primitivismo. E todos certos de sua importância, incapazes de qualquer
lance de autoironia. Grotescos.

%\textbf{Gianni:}
\itshape
 Segundo Lula, a direita está tentando criar um novo
candidato, visto que não há nenhum.

%\textbf{Mino:} 
\normalfont
Essa tentativa existe. A questão é a seguinte. Os
candidatos existem em função de Lula. Então, neste momento, o candidato
mais forte depois de Lula é Jair Bolsonaro, que tem quase um terço das
intenções de voto de Lula. Mas ele é o mais forte na oposição. Os
golpistas não têm candidato. Daí minha dúvida: haverá eleições ou
teremos um golpe dentro do golpe com ares de legalidade?

%\textbf{Gianni:}
\itshape
 E quanto a Bolsonaro?

%\textbf{Mino:} 
\normalfont
Bolsonaro é o impecável representante de um Brasil
retrógrado. Se ele fosse presidente, instituiria o ensino do
criacionismo nas escolas. No tempo da ditadura ensinavam Moral e Cívica,
com Bolsonaro teríamos o criacionismo. Delírio absoluto. A recente
exposição de quadros eróticos no Masp em que foi proibida a entrada de
menores de 18 anos foi uma aberração. É como se colocassem na porta da
Capela Sistina, em Roma, onde na abóbada, entre outros, aparece um Adão
nu, uma placa também proibindo a entrada de menores de 18 anos.

%\textbf{Gianni:}
\itshape
 Essa mentalidade explicaria a ascensão de Bolsonaro?

%\textbf{Mino:} 
\normalfont
Falam no fascismo de Bolsonaro, mas não é por aí.
Bolsonaro representa antes de mais nada o atraso cultural e intelectual,
e não é por acaso que conta com o apoio dos evangélicos. Como candidato,
no entanto, acredito que perca força se Lula for condenado em segunda
instância.

%\textbf{Gianni:}
\itshape
 E os outros candidatos?

%\textbf{Mino:} 
\normalfont
Como já disse, as quadrilhas não emplacariam qualquer um
dos cogitados, nem mesmo Geraldo Alckmin. Fala"-se em Luciano Huck creio,
porém, que se trate de puro humorismo. Um candidato respeitável é Ciro
Gomes, mas ainda não conseguiu uma dimensão nacional. Há também a Marina
Silva, que, a meu ver, na hora do pleito mostraria escassa força nas
urnas. Ela não tem opinião a respeito de coisa alguma.

%\textbf{Gianni:}
\itshape
 De qualquer forma, o eleitorado paulista não votaria em
Lula. De onde vem tanto reacionarismo?

%\textbf{Mino:} 
\normalfont
São Paulo era a locomotiva, como já se disse, movida a
café. Mandavam os comissários do café. Alvorecia, contudo, um começo de
industrialização. Levas de imigrantes chegaram também à capital, e as
greves de 1907, 1909 e 1917 foram importantes. Os organizadores foram
400 anarquistas. Altino Arantes, já governador de São Paulo, os
deportou. Mandou"-os de volta para a Itália. Outra imigração positiva
como a japonesa não representava entrave político. Com os anarquistas o
operariado começou a se manifestar. Mas, quando foram deportados, o
reacionarismo afundou suas raízes. Culminou com a Revolução de 1932.
Eles chamam de revolução um movimento separatista.

%\textbf{Gianni:}
\itshape
 Esse ódio ao imigrante por parte dos chamados
``quatrocentões'' não pode ter sido projetado contra o nordestino e, no
caso, contra Lula?

%\textbf{Mino:} 
\normalfont
Durou muito tempo esse ódio ao imigrante. Mas contra Lula
diminuiu. Tomou outra forma. Encaixou"-se no ódio de classe. Antes de
Lula, a repulsa ao nordestino foi bem mais forte. Não houve políticas
para segurar os nordestinos nas terras deles. Nunca houve. Teria sido
necessário oferecer"-lhes condições melhores de vida e desenvolvimento.

%\textbf{Gianni:}
\itshape
 Do ponto de vista jornalístico, é importante entender
como funcionaria um movimento de reação do povo. A chamada ``Primavera
Árabe'', uma invenção do Ocidente que não resultou em nada significante,
foi organizada através das redes sociais. Donald Trump venceu no \versal{EUA}
bastante ajudado pela Rússia, que teria produzido 80 mil \emph{posts}
para atingir 126 milhões de pessoas nas redes sociais, segundo a \versal{BBC}. O
próprio Lula tem uma participação importante nas redes sociais. No
entanto, os jornalões e as revistas reacionárias ainda dão as cartas. No
último ano, disse Lula, saíram 56 capas de revista contra ele.

%\textbf{Mino:} 
\normalfont
A mídia impressa, de fato, teve muito êxito na sua
campanha anti-Lula e anti-\versal{PT}. Mas acho que funciona mais ainda com a
televisão aberta, sobretudo com a Globo, que alcança os pobres. Pobre
não tem acesso à \versal{TV} a cabo e, portanto, assiste aos canais
convencionais. A televisão também está toda contra Lula.

%\textbf{Gianni:}
\itshape
 Entre as pessoas mais ricas do Brasil encontram"-se os
três irmãos Marinho, filhos do suposto jornalista Roberto Marinho.
Teriam entre eles, os três irmãos, uma fortuna estimada em 11,3 bilhões
de dólares. Segundo Marcos Coimbra, presidente do Instituto Vox Populi,
Roberto Marinho engoliu o ``Sapo Barbudo'', achando que ele não teria
fôlego na política. Não foi o caso. E agora os sucessores dele querem
destruir a imagem pessoal e política do ex"-presidente. Lula me disse que
no último ano houve 20 horas de Jornal da Globo contra ele. A Globo,
sabemos, fez campanha também contra Temer, mas o que querem mesmo é a
pele de Lula.

%\textbf{Mino:} 
\normalfont
Hipocrisia à enésima potência, fingem"-se de moralistas.
Vamos entender claramente. Os Marinho querem colocar no poder um homem
deles.

%\textbf{Gianni:}
\itshape
 É estranho a mídia internacional, salvo raras exceções
na Europa, estar silenciosa a respeito do golpe contra Dilma e o
julgamento ilegal de Lula.

%\textbf{Mino:} 
\normalfont
Em primeiro lugar, o Brasil é muito distante. E não é
fácil para o europeu entender o que está acontecendo por aqui.
Ironicamente, isso prova que a casa"-grande está de pé, e o Brasil na
Idade Média. A mídia europeia, como você diz, não entende isso. Veja o
quadro: um golpe perpetrado pela aliança entre Executivo, Legislativo,
Judiciário e a mídia. E jogando no lixo a Constituição de 1988, que,
certa ou errada, perfeita ou não, completa ou não, é a Constituição. E,
ainda, há setores da polícia federal transformados em jagunços da
casa"-grande. Explique isso para um europeu. Ele não tem condições de
entender uma situação dessas.

%\textbf{Gianni:}
\itshape
 O \emph{Financial Times} e o \emph{Wall Street
Journal}, por exemplo, estão felizes com a ``estabilidade econômica''
sob Temer. E, por tabela, aplaudem a privatização da Petrobras. É a
ditadura do capital?

%\textbf{Mino:} 
\normalfont
É a ditadura do capital. O neoliberalismo quer isso. É um
desastre mundial absoluto. E no Brasil assume proporções gigantescas.
Isso porque o golpe favorece somente os ricos e os super"-ricos. Qual a
saída? A reação popular maciça e destemida. E bem liderada. A tomada da
casa"-grande. Mas eu não me iludo, esta é a única solução, no meu
entendimento, impossível, nas condições em que o País se encontra, com
fortíssimas raízes no passado colonial e escravocrata.

%\textbf{Gianni:}
\itshape
 Lula seria ainda o único a poder comandar o País com
sucesso?

%\textbf{Mino:} 
\normalfont
Quando governou o Brasil, Lula foi uma estrela
internacional. Em 2009, a prestigiosa revista \emph{Foreign Policy}, dos
\versal{EUA}, disse que Celso Amorim era o chanceler mais importante do mundo.
Acho que a atual falta de lideranças depende muito dessa ausência da
política, que Lula com razão aponta. Por conta disso, é difícil que
surjam lideranças se a política sofre. Mas chamo atenção para lideranças
que estão nascendo no mundo. O Corbyn e uma liderança forte no Reino
Unido. Idem o Mélénchon, na França. O jovem italiano Speranza, de 40
anos, é uma liderança nascente. Esta situação, na qual o mundo todo se
sujeita à força do capital, provoca o surgimento de novas lideranças. E
surgirão outros e outras que tomarão a defesa do Estado Democrático de
Direito. Inevitavelmente.

%\textbf{Gianni:}
\itshape
 Donald Trump não gostaria de uma vitória de Lula, mas
Barack Obama e o ex"-presidente brasileiro se davam bem.

%\textbf{Mino:} 
\normalfont
Se davam bem. Mas a postura muito independente de Lula
era difícil de ser apreciada também por Obama. A construção dos \versal{BRICS} e
outras situações não poderiam agradar aos Estados Unidos. Lula, outro
exemplo, foi a Israel e tomou uma posição muito clara pró-Palestina. E
ao mesmo tempo tudo aquilo que se fez em termos de política
internacional foi impecável.

%\textbf{Gianni:}
\itshape
 Celso Amorim me disse em entrevista recente que, antes
de Lula, o Brasil jogava na Segunda Divisão, mesmo sendo o primeiro
colocado nesta Série B. Mas sob Lula foi promovido à Série A. E fizeram
isso. Amorim disse que a política externa de Lula na América Latina, na
África e no Oriente Médio incomodou os países ricos. Entre outros feitos
houve as negociações que aproximaram o Ocidente do Irã, em 2010,
capitaneadas por Brasil e Turquia. Graças a esse pré"-acordo, copiado
depois pelos Estados Unidos, temos hoje negociações entre a chamada
``comunidade internacional'' e Teerã. Claro que, hoje, Trump está
tentando acabar com tudo isso.

%\textbf{Mino:} 
\normalfont
Depois de concluído com sucesso o pré"-acordo, os Estados
Unidos não acharam graça. Houvera, neste caso, uma carta de Obama
respondendo a Lula e autorizando"-o a agir, ou pelo menos louvando a
iniciativa. Essa carta precedeu a ida a Teerã.

%\textbf{Gianni:}
\itshape
 Por sua vez, Fernando Henrique Cardoso era submisso aos
\versal{EUA}.

%\textbf{Mino:} 
\normalfont
Completamente. Abraçava Bill Clinton a toda hora.

%\textbf{Gianni:}
\itshape
 Voltamos ao poder dos mercados, o que explicaria a
submissão de Fernando Henrique aos países ricos.

%\textbf{Mino:} 
\normalfont
A privatização, quero dizer a roubalheira sob Fernando
Henrique, foi aplaudida pelos países da chamada ``comunidade
internacional''. Agora vem aí a privatização da Petrobras. E o que foi a
licitação do pré"-sal? Quando se tira o pré"-sal da Petrobras, você
vulnera fatalmente seu valor, de sorte a vendê"-la a preço de banana.
Aliás, o Fernando Henrique presidente queria privatizar a Petrobras.

%\textbf{Gianni:}
\itshape
 Objetivo concretizado sob Temer.

%\textbf{Mino:} 
\normalfont
É a grande vitória tardia de Fernando Henrique. E por que
um professor universitário, aposentado, tem um apartamento de nababo em
São Paulo, situado em bairro nobre, como se diz aqui, e uma fazenda de
muitos alqueires. Não estamos falando em um ridículo triplex na Praia
das Astúrias, lugar de farofeiros no Guarujá. Ou de um sítio em Atibaia
com vista para a favela. Os social"-democratas brasileiros não são os
social"-democratas de Willy Brandt. Eles são o esteio da extrema"-direita,
são o instrumento mais atilado da casa"-grande. Envolvidos até o talo no
golpe.

%\textbf{Gianni:}
\itshape
 Causa estranheza o fato de os críticos de Lula sempre
dizerem que Fernando Henrique promoveu a economia do País.

%\textbf{Mino:} 
\normalfont
Referem"-se à estabilidade da moeda. De fato, os
economistas que cercavam \versal{FHC}, inspirados por uma manobra israelense de
muito tempo atrás, de valorizar a moeda para torná"-la mais segura,
conseguiram interromper a queda do cruzeiro ao transformá"-lo em real.
Isso já se deu com Itamar Franco, substituto de Fernando Collor, quando
Fernando Henrique era ministro do Exterior. A operação começou pela \versal{URV}.
Aos fatos. Durante o primeiro mandato de Fernando Henrique, houve o
\emph{crack} russo e o Brasil quebrou. Para conseguir sua reeleição,
Fernando Henrique comprou votos, como é do conhecimento até do mundo
mineral. Era necessária a maioria de dois terços para a mudança
constitucional e ele fez uma campanha eleitoral com a sombra da bandeira
da estabilidade. Roberto Marinho acreditava, instigado por Miriam
Leitão. Doze dias após a posse para o segundo mandato, ou seja, já
reeleito, no dia 12 de janeiro de 1999, Fernando Henrique desvalorizou o
real. Abriu um rombo na Globo e quebrou o Brasil novamente. Quando Lula
chegou ao poder, os cofres do Estado estavam vazios. O Brasil devia mais
de 200 bilhões de dólares. O Lula pagou tudo. E mais: encheu as burras
do Estado. Colocou muita grana nos cofres. O Estado tinha lastro forte
mesmo. Como podem tecer elogios sobre a política econômica do Fernando
Henrique? Ele quebrou o Brasil. E a privatização das comunicações foi a
maior bandalheira de todos os tempos.

%\textbf{Gianni:}
\itshape
 Agora, a roubalheira é mais transparente. Vemos pessoas
correndo com malas repletas de dinheiro. E você falou sobre os policiais
federais que se comportam como jagunços da casa"-grande.

%\textbf{Mino:} 
\normalfont
Largos setores da polícia federal aderiram ao golpe.
Apoiaram o \emph{impeachment} de Dilma e ofereceram proteção aos
golpistas. E estão prontos a se engajar em qualquer tipo de operação.

%\textbf{Gianni:}
\itshape
 Um oficial do Exército já se exprimiu em uma casa
maçônica sobre uma possível intervenção.

%\textbf{Mino:} 
\normalfont
Não acredito em intervenção militar, mas não sei como os
fardados se comportariam se o povo saísse às ruas para protestar e
disposto a brigar. Não sei como eles se comportariam em caso de
desordem. Seria uma incógnita muito séria, mas não cogitemos de algo
impossível. Creio que os militares, salvo uma ou outra exceção, têm se
portado muito bem, dentro da Constituição.

%\textbf{Gianni:}
\itshape
 Por outro lado, os militares poderiam se voltar contra
os quadrilheiros, defender a Petrobras, por exemplo.

%\textbf{Mino:} 
\normalfont
As Forças Armadas são nacionalistas. Essa história do
pré"-sal certamente os irrita muito. Por ora, respeitam, como disse, a
Constituição. Quem não a respeita são os Três Poderes da República.


%\textbf{Gianni:}
\itshape
No golpe de
1964, houve tanques para derrubar João Goulart.

%\textbf{Mino:} 
\normalfont
Em volta de Jango havia esquerdistas sinceros. As
chamadas reformas de base, a plataforma de Goulart, eram coisa séria.
Era exatamente taxar os ricos, reforma agrária em profundidade, e assim
por diante. Era um governo de esquerda. E havia gente de esquerda. Não
digo comunista. De esquerda no sentido de reformador do País, com o
objetivo de liquidar o abismo que separa ricos e pobres. Queriam acabar
com a casa"-grande e a senzala.

%\textbf{Gianni:}
\itshape
 Mas Washington viu a situação de outro ângulo.

%\textbf{Mino:} 
\normalfont
Os Estados Unidos ofereciam até suporte bélico. John
Kennedy foi um desastre.

%\textbf{Gianni:}
\itshape
 Em 1964, houve o apoio das elites para derrubar o
governo legítimo. No golpe de 2016, sem tanques, o mercado quis dominar
o Estado. Esse último golpe me parece mais sutil.

%\textbf{Mino:} 
\normalfont
O golpe de 2016 foi diferente e adequado ao contexto
político no Brasil e no mundo. Mas veja: o de 1964 foi um golpe
desfechado pelos militares, chamados pelas elites, haja vista os
editoriais do \emph{Estadão} na época, ou do \emph{Globo}, ou do
\emph{Jornal do} \emph{Brasil}. Os militares foram chamados pela
casa"-grande para executar o serviço sujo. Depois, talvez os militares
tenham gostado muito de estar lá em cima. Na verdade, o golpe de 1964
foi civil e militar. Porque a deixa saiu das elites. O povo não soube
como reagir, mas começou a medrar uma espécie de resistência que depois
se fortaleceu. Por parte dos ditadores havia aquela hipocrisia de manter
um sistema eleitoral. Finalmente, o \versal{AI}-5, em dezembro de 1968, confirmou
com clareza o golpe. Os militares ganharam então plenos poderes. Com o
golpe dentro do golpe o Congresso foi fechado, para reabrir depois de
algum tempo. Houve cassações aos magotes, mas nem por isso a oposição
arrefeceu. E o \versal{MDB}, comandado pelo Ulysses Guimarães, aglutinou todos os
opositores. Alguns partiram para a luta armada e, no caso, morreram mais
de 400. Milhares foram torturados. Este é o resultado de uma tentativa
de luta armada. A qual, diga"-se, os comunistas rejeitaram. De qualquer
maneira, a ditadura foi um episódio que deveria ter fortalecido o anseio
pela democracia. Naquele tempo, mesmo perseguido pela censura, eu tinha
grandes esperanças de que chegaríamos finalmente a encontrar o caminho
de uma vez por todas. O Brasil havia atravessado um período econômico
muito favorável, de Getúlio Vargas a Jango Goulart. Foi quando o País
chegou a ser a décima quinta potência industrial do mundo. O que talvez
até explique por que acabou surgindo, já em tempos de ditadura, uma nova
geração de líderes sindicais. De alguma forma, estava se criando, ainda
que de forma muito confusa, incerta, um proletariado. O proletariado
sempre foi a bucha de canhão da esquerda. Para mim, foi um período de
esperança.

%\textbf{Gianni:}
\itshape
 Inclusive, do ponto de vista jornalístico?

%\textbf{Mino:} 
\normalfont
As minhas esperanças não chegavam a tanto. A chamada
grande mídia, com exceção de \emph{Veja}, que eu dirigia, apoiava a
ditadura. De todo modo, a história da censura é contada de uma maneira
realmente singular. O \emph{Estadão} foi censurado porque os dirigentes
do jornal queriam Carlos Lacerda na Presidência. No entanto, Lacerda
acabou cassado ao criar uma Frente Ampla com Juscelino Kubitschek e João
Goulart. Foi assim que teve início a censura contra o \emph{Estadão}. A
censura era feita na redação do jornal e os buracos deixados pelas
tesouras dos censores eram preenchidos com versos de Camões. No
\emph{Jornal da Tarde} publicavam receitas de bolo. Os outros jornalões
não foram censurados. Quem foi censurado realmente? \emph{Veja} e os
alternativos. E censura feroz, muito feroz.

%\textbf{Gianni:}
\itshape
 Quando, com o fim da ditadura, suas esperanças
arrefeceram?

%\textbf{Mino:} 
\normalfont
Logo que começou a chamada redemocratização, percebi que
tinha me enganado redondamente.

%\textbf{Gianni:}
\itshape
 A partir de 1985?

%\textbf{Mino:} 
\normalfont
Desde o momento em que Figueiredo sai pela porta dos
fundos e fomos para as eleições indiretas. Aliás, Golbery deixara o
governo depois das bombas patéticas do Rio Centro, mas o plano dele foi
cumprido até o fim: Tancredo \emph{versus} Maluf. O que Golbery não
podia prever foi a morte de Tancredo. E a Presidência ficou com quem
derrubara no Congresso a emenda das Diretas Já. A emenda ficou muito
pior que o soneto, o resultado foi um escárnio.

%\textbf{Gianni:}
\itshape
 Quando você começou a se interessar por Lula?

%\textbf{Mino:} 
\normalfont
Final de 1977, começo do ano seguinte, quando se
preparava a greve de 1978. Um repórter da revista \emph{IstoÉ}, Bernardo
Lerer, irmão do David Lerer, que vinha do \versal{PTB} de Alberto Pasqualini e
Leonel Brizola. Bernardo me disse: ``Olha, esse Lula é muito
interessante. Valeria a pena falar com ele''. Lá fomos nós, o Bernardo
Lerer e eu, ao Sindicato dos Metalúrgicos de São Bernardo e Diadema. Ao
entrar no prédio, logo notei uma reprodução do quadro de Pelizza da
Volpedo, \emph{O Quarto Poder}, título que não se referia à imprensa,
mas aos trabalhadores. O que mostrava a tela? Um senhor enchapelado,
barbudo, caminhando na frente do povo. E aí Lula vem ao meu encontro.
Parecia que tinha saído do quadro. Faltava o chapéu. Descrevi essa cena
no meu livro, \emph{O Brasil}. Lula e eu tivemos uma longa conversa e
fiquei muito impressionado. A primeira coisa que me impressiona em Lula
é o seu senso de humor, característica muito importante. Sabe lidar com
a ironia. E tem definições muito agudas em relação a vários temas. Foi
uma conversa excelente. Falamos sobre a vida dele desde o começo. Desde
a viagem de caminhão de Garanhuns a São Paulo, sobre a mãe faxineira que
aqui em São Paulo cuidava de oito filhos, e do único irmão que se
interessou por política, Frei Chico, comunista que foi do \versal{PC}doB.

%\textbf{Gianni:}
\itshape
 Segundo Lula, o Frei Chico não teve influência
ideológica. O que o irmão fez foi conseguir para ele um posto no
sindicato.

%\textbf{Mino:} 
\normalfont
É isso mesmo. Mas, de volta ao dia em que conheci Lula,
Bernardo Lerer e eu decidimos fazer uma entrevista no formato pergunta e
resposta com Lula. O texto da reportagem eu escrevi. E fizemos a famosa
capa, concebida pelo capista Hélio de Almeida e publicada em 9 de
fevereiro de 1978. A greve seria realizada em abril daquele ano, quando
vencia o contrato coletivo de trabalho.

%\textbf{Gianni:}
\itshape
 Qual foi a reação à capa?

%\textbf{Mino:} 
\normalfont
A reação dos leitores me pareceu muito boa, a
\emph{IstoÉ} era um sucesso de vendas. Ela chegou a 100 mil exemplares
de tiragem, em um tempo em que a de \emph{Veja} era de 200 mil. Quer
dizer, era uma concorrente real. E os ditadores, ao contrário deste
atual governo golpista encabeçado por quadrilheiros, não vetavam
anúncios. Nesse tempo, a censura tinha terminado. A censura terminou, de
fato, no começo de 1977, quando a revista se tornou semanal exatamente
por causa disso.

%\textbf{Gianni:}
\itshape
 E quem foi o homem por trás do fim da censura?

%\textbf{Mino:} 
\normalfont
O general Golbery do Couto e Silva. Ele tinha, de fato, o
plano de retorno à normalidade democrática, segundo a ideia dele de
democracia.

%\textbf{Gianni:}
\itshape
 Você disse que o general Golbery é o autor da reforma
partidária de 1979, com o objetivo de estilhaçar o \versal{MDB}. Mas Lula não
acredita que foi Golbery quem facilitou a formação do \versal{PT}.

%\textbf{Mino:} 
\normalfont
Eu nunca disse que Golbery facilitou a formação do \versal{PT}.
Disse que, ao estilhaçar o \versal{MDB}, ele precipitou o nascimento do \versal{PT}. O
chefe da Casa Civil de Geisel entendia que Lula era diferente dos
pelegos. Tratava"-se, além do mais, de um sindicalista de \versal{QI} alto. Para
Golbery, Lula tinha um pensamento não marcado pelo marxismo, não marcado
por uma postura esquerdista tradicional, digamos. Golbery sabia do meu
respeito e de minha amizade por Lula e me perguntou a respeito. E foi
então que mandou emissários para ouvir Lula no Dops. E que perguntaram?
``Como eu via a vida -- diz Lula --, o que pensava do mundo, como
enxergava o trabalho no Brasil, o que achava dos empresários. Perguntas
desse tipo.'' Os emissários, engravatados, chegavam a mando do
``cacique''. Ao sair da prisão, Lula me disse: ``Mas quem será o
cacique?'' Passei muito tempo perguntando ao ex"-carcereiro Romeu Tuma
quem seria o cacique, e ele não respondia. Enfim, pouco antes de morrer,
confessou: ``Era o Golbery''. Tuma gostava de Lula, e foi um carcereiro
suave.

%\textbf{Gianni:}
\itshape
 Houve então uma tentativa do Golbery de estilhaçar as
esquerdas, como fez François Mitterrand com as direitas n França nos
anos 1980?

%\textbf{Mino:} 
\normalfont
Tancredo Neves e o próprio Ulysses Guimarães eram
conservadores iluminados. Tancredo, embora subisse nos palanques das
``Diretas Já'', queria as indiretas. Ulysses Guimarães foi o ``Senhor
Diretas'', e ganharia se diretas fossem. Havia ao redor dele paspalhos
como Fernando Henrique Cardoso, que se dizia de esquerda, ou como o José
Serra. E estavam todos no \versal{MDB}. André Franco Montoro também fazia parte
desse grupo, ótima pessoa, democrata"-cristão à velha moda. A turma da
oposição estava toda no \versal{MDB} do Doutor Ulysses. Vamos deixar bem claro:
não era uma esquerda. Havia uma minoria de esquerda, inclusive
marxistas. Tratava"-se, em todo caso, de uma oposição valente. Donde a
ditadura ter recorrido a inúmeras cassações. Deixava o Congresso aberto,
mas caçava os congressistas quando bem entendesse. Houve o famoso pacote
de abril, em 1977, começo da \emph{IstoÉ} semanal.

%\textbf{Gianni:}
\itshape
 Como se deu o seu contato com Golbery?

%\textbf{Mino:} 
\normalfont
Quem tinha contato com ele há algum tempo era Elio
Gaspari, quando ele trabalhava na \emph{Veja}. Elio foi editor de
política na revista até 1974. Um dia, me disse: ``Você precisa conhecer
Golbery, uma figura muito interessante''. Estamos no início dos anos
1970. Golbery estava fora do governo, pertencia ao grupo dos
Sorbonianos…

%\textbf{Gianni:}
\itshape
 Mas nenhum deles tinha elos com a Sorbonne.

%\textbf{Mino:} 
\normalfont
O primeiro foi Castello Branco, que tinha lido muito
Victor Hugo e isso o qualificava como Sorboniano. Não conheci
pessoalmente o Castello Branco, mas receio que fosse um medíocre. Mas
não importa. O fato é que conheci Golbery, no Rio de Janeiro, em 1972,
quando já era diretor de \emph{Veja} havia quatro anos. Naquele momento
Golbery era presidente da Dow Chemical no Brasil. Golbery era cordial,
uma pessoa muito simpática. Acabamos amigos, apesar das óbvias
diferenças ideológicas. E assim tornou"-se hábito eu ir ao Rio visitá"-lo,
ainda como presidente da Dow Chemical. Íamos almoçar em uma dessas
típicas padarias portuguesas populares do Rio, onde se comia um bom
bacalhau, por exemplo. Ele foi uma fonte fundamental. Como disse,
tínhamos ideias opostas. Ele era um filho da Guerra Fria, gostava,
porém, de ler tudo o que lhe recomendava, a edição domingueira do
\emph{New York Times}, de direita e de esquerda, e de artes plásticas.
Um ponto em comum. Tenho certeza de que ele simpatizou comigo. Falava
muito bem de mim.

%\textbf{Gianni:}
\itshape
 À época, Golbery já tinha uma estratégia para devolver
o governo aos civis?

%\textbf{Mino:} 
\normalfont
Estava tramando. Havia começado, pouco a pouco, e
pretendia devolver tudo aos civis. De saída, criou o contato com a
casa"-grande. Mas veja: era um homem em perfeita sintonia com o contexto
da Guerra Fria, portanto, com a geopolítica dominada pelos \versal{EUA}. Por isso
acabou na Dow Chemical. Mas, ao mesmo tempo, achava que se devia colocar
a casa em ordem. Para ele, havia uma grande bagunça em toda a América
Latina, por conta de Fidel Castro. Ele era, antes de tudo, um crente da
divisão maniqueísta do mundo. Era também um estrategista notável, odiado
pelos militares. Foi ele quem orientou a sucessão de Médici por Geisel,
que aos olhos dos militares levava a enorme vantagem de ser irmão de
outro general, Orlando Geisel.

%\textbf{Gianni:}
\itshape
 Por que vantagem?

%\textbf{Mino:} 
\normalfont
Porque Orlando Geisel era uma espécie de condestável do
poder ditatorial. De garantia. Mas, ao mesmo tempo, Golbery, homem muito
sagaz, percebia que Geisel era o títere ideal para o titereiro Golbery.
Na verdade, o titereiro nunca me disse que Geisel era o títere, mas a
situação ficou muito clara, clara demais, ao menos para mim. Golbery
sabia como dobrar o homem e conduzi"-lo pelo caminho que havia traçado.
Ao aceitar Geisel como seu herdeiro, Médici impôs uma condição: Golbery
não haveria de figurar no novo governo. Geisel tornou"-se o ditador da
vez e chamou Golbery para a chefia da Casa Civil, em 1974. A relação
entre Golbery e Geisel era bastante formal, tratavam"-se de senhor. E
Golbery sabia que conseguiria levar Geisel na conversa. Qual era o plano
do Golbery? Partir para uma anistia incompleta, mas anistia. Esses
seriam os dois passos finais de Geisel: anistia e reforma partidária.
Anistia no começo de 1979 e a reforma partidária no fim de 1979. Aí a
coisa marchou. Golbery foi um Mago Merlin da ditadura.

%\textbf{Gianni:}
\itshape
 Em 1974, você levou a Manuela e eu a Brasília, quando
foi se encontrar com Golbery.

%\textbf{Mino:} 
\normalfont
Espere aí, vou pegar \emph{O Brasil}, meu livro editado
pela Record em 2014 e vou ler a passagem em que conto esta história,
vivida com vocês, no quarto andar do Palácio da Alvorada. ``Ao chegar ao
gabinete do chefe da Casa Civil demos com uma presença inesperada na
antessala: Roberto Civita. Arregalei os olhos, ele explicou que o pai o
informou a respeito da minha ida à capital para discutir a volta da
censura depois de três semanas de liberdade. Eu negociara a saída dos
censores com o futuro ministro da Justiça, Armando Falcão, ainda durante
a gestação do governo Geisel, em fevereiro de 1974. Dois meses depois,
Falcão me chamou a Brasília para dizer que os censores estavam de saída.
Eu sublinhei: sem compromisso algum de nossa parte. Claro, disse ele, e
me deu um livro da sua lavra, intitulado A \emph{Revolução Permanente}.
Passei então pelo gabinete de Golbery e ele sentenciou: `Falcão é o
nosso Trotski'''.

``Passo a ler, a partir do momento em que interpelo Roberto
Civita… Ah, sim, durante as três semanas sem censura não tínhamos
nos comportado como os militares pretendiam, sobretudo na terceira, com
a publicação de uma charge de Millôr Fernandes. Leio: `Você não pediu
audiência, não tem hora', proclamo. Ele insiste, à beira da imploração.
O meu tom chama a atenção de Manuela e Gianni, que encaram a cena sem
entender o assunto, percebem, porém, que o pai está muito irritado,
enquanto o outro tem jeito de pedinte. Lurdinha (\emph{secretária de
Golbery, adorável pessoa}) traz uma laranjada para as crianças e avisa
que o general está à espera. Admito: `Você entra comigo, mas se
compromete a não abrir a boca'. Ele promete. Na conversa que se segue no
gabinete da Casa Civil, o meu argumento é óbvio: \emph{Veja} é uma
revista semanal que encerra o trabalho na noite de sábado e vai às
bancas às segundas"-feiras, obrigá"-la a submeter textos e fotos aos
censores na terça significa inviabilizá"-la. Pergunto a Golbery: `Os
senhores pretendem que \emph{Veja} simplesmente acabe?' Não, nada disso.
`Então é preciso pôr em prática outro sistema.' O chefe da Casa Civil
entende e concorda. Diz: `Vá até o Ministério da Justiça, fale com
Falcão, a Lurdinha já vai avisá"-lo, diga a ele que vamos procurar uma
saída até amanhã no máximo, a próxima edição tem de sair regularmente'.
Golbery fica de pé, hora da despedida. O general não conhecia o
patrãozinho que até aquele momento cumpriu a promessa feita na
antessala. E de supetão abre a boca: `General, se o senhor acha que
devemos tomar alguma providência em relação ao Millôr Fernandes…'
Golbery fulminou"-o: `Senhor Civita, não pedi a cabeça de ninguém'. Há
quem diga que Arci se parece com um cachorrão pelancudo, nunca como
nesse instante a semelhança me pareceu tão evidente. Como cão enxotado
do quarto sai de focinho a fazer cócegas ao carpete. Na antessala digo,
com irrecorrível desprezo: `Bem que tinha pedido que você ficasse
calado, mas você é um imbecil'''.

%\textbf{Gianni:}
\itshape
 Lembro da cena. E como continua o plano de Golbery de
entregar o poder aos civis?

%\textbf{Mino:} 
\normalfont
Em 1974, Orestes Quércia elegeu"-se senador. O \versal{MDB} colocou
deputados e senadores em todos os cantos. Houve cassações às pamparras.
Fui ter com o Golbery e ele disse: ``Olha, eu sou muito parlapatão, mas
veja, não tem jeito. Precisamos fazer uma reforma partidária. Quem é de
esquerda que fique à esquerda, e quem é de direita que fique à direita.
Para deixar tudo mais claro''. No ano seguinte, 1975, Golbery teve um
descolamento da retina. Foi operado na Espanha, onde na época havia os
bambas desse problema. Golbery voltou ao Rio de Janeiro. Ficou entrevado
em um quarto, onde passou um mês inteiro na escuridão. Finalmente,
retornou a Brasília. No dia 3 de abril, aproveitando"-se da ausência de
Golbery, Geisel havia pronunciado o discurso da pá de cal, ou seja,
estava encerrada a distensão lenta e gradual, porém segura. Fui visitar
Golbery no dia seguinte ao pronunciamento de Geisel. Jazia sobre a mesa
o discurso do ditador. Golbery me disse: ``Sabe quem escreveu isso aqui?
O João Paulo dos Reis Velloso''. Reis Velloso era então ministro do
Planejamento. O discurso estava sublinhado em amarelo em diversas
passagens. ``Isso vai provocar o retorno do terror de Estado. Vai ter
cousas do arco"-da"-velha. É uma estupidez total. Cuide"-se'', avisou
Golbery. Eu senti nitidamente que ele via em Geisel um desastrado. A
perseguição desencadeada culminou com a morte de Vlado Herzog, em
outubro de 1975. Em agosto, Golbery também me disse: ``Olha, se o Geisel
não ficar atento, ele não vai fazer o sucessor dele. E o próximo
presidente será o Frota''. O ministro do Exército. E aí?, indaguei. ``O
Frota tem que cair.'' Quando? Já? Amanhã? ``Até o dia 12 de outubro de
1977'', respondeu Golbery. Estávamos em agosto de 1975. Frota caiu
exatamente no dia 12 de outubro de 1977. Dois anos depois, Geisel fez a
anistia, bastante parcial. Aquela anistia foi aprovada pela chamada
Comissão da Verdade, pelo Congresso e pelo Supremo Tribunal Federal, em
adiantados tempos de redemocratização e lá estava Paulo Sérgio Pinheiro,
um dos grandes hipócritas que conheci. E Golbery conduziu a reforma
partidária em 1979. Tancredo formou imediatamente o Partido Popular,
Lula o Partido dos Trabalhadores, Brizola, de quem tinham roubado o
Partido Trabalhista Brasileiro, fez o \versal{PDT}. E o Doutor Ulysses ficou com
o \versal{PMDB}. Depois das bombas do Rio Centro, em 1981, Golbery disse a
Figueiredo: ``E preciso demitir o general Gentil Marcondes''. Marcondes
era então o comandante do I Exército e, portanto, o primeiro mandante
das bombas. Figueiredo rebateu: ``Não, não, vamos chamar o Octávio''.
Veio o general Octávio Medeiros, a quem Golbery uma vez me apresentou
como Camarada Dimitrov, a lembrar ironicamente meu nome de batismo,
Demetrio. O chefe da Casa Civil tinha muito senso de humor, Medeiros nem
um pouco. E, claro, achava que Marcondes não deveria ser demitido. De
todo modo, o então chefe do \versal{SNI} sonhava ser o sexto ditador e Figueiredo
apreciava a ideia. Não sei se na ocasião Golbery cometeu um erro ou agiu
de caso pensado. E Figueiredo sentenciou: ``Então fico com o Medeiros''.

%\textbf{Gianni:}
\itshape
 Quando Golbery disse para você se cuidar após o
pronunciamento de Geisel, em abril de 1975, ele sabia que pediriam a sua
cabeça…

%\textbf{Mino:} 
\normalfont
Claro. Para os Civita estava em jogo o empréstimo pedido
à Caixa Econômica Federal, então presidida por Karlos Rischbieter.
Tratava"-se de um empréstimo de 50 milhões de dólares. De fato, quando
chegou a minha hora por causa desse empréstimo, Rischbieter teve de
entregar a aprovação a um superior hierárquico, pois se tratava de uma
questão política A pasta empacou na mesa do ministro da Justiça, Armando
Falcão.

%\textbf{Gianni:}
\itshape
 Ele disse que você era o mais chato de todos os
jornalistas.

%\textbf{Mino:} 
\normalfont
Não, esse foi o Figueiredo, e me fez um grande elogio ao
me confundir com Roberto Marinho e Victor Civita. Figueiredo disse:
``Esses só vêm me pedir favores. O Mino não pede coisa alguma. É um
chato. Geisel o detestava. Mino reescreveria os Evangelhos, mas não tem
o rabo preso''. Foi o maior elogio que já recebi. Mas ele me confundia
com donos de empresas. De qualquer modo, quando, em fevereiro de 1976,
Falcão, em nome de Geisel, pediu a minha cabeça em troca do empréstimo,
eu me demiti para não receber um único, escasso tostão dos Civita.
Anotei, contudo, que fui muito mais caro do que Jesus Cristo, vendido
por 30 dinheiros. Golbery não mexeu uma palha quando fui embora da
\emph{Veja}. Nas suas memórias, Karlos Rischbieter conta que foi ter com
ele e disse: ``Mas e o Mino?'' E Golbery respondeu: ``O Geisel o odeia.
Não tem jeito, não posso fazer nada, não posso defender o Mino''.
Estavam em jogo as razões de Estado. Fui trabalhar com meu irmão Luis e
com Domingo Alzugaray em uma revista mensal que se chamou \emph{IstoÉ.}
Anódina, inodora e mensal. A primeira capa foi o Beethoven, o assunto
era a surdez. Como já disse, \emph{IstoÉ} virou semanal quando acabou a
censura, em 1977. O último a ser censurado foi o jornal \emph{O São
Paulo}, da Cúria Metropolitana. Dom Paulo Evaristo Arns era outra pessoa
detestada por Geisel. Grande e santa figura o Dom Paulo daqueles tempos
turvos. Quando já tinha passado, ele me contou que ao levar a Golbery,
chefe da Casa Civil, uma lista de desaparecidos, o general não escondeu
as lágrimas. Publiquei esta história na \emph{CartaCapital}, ainda
mensal.

%\textbf{Gianni:}
\itshape
 E com a \emph{IstoÉ} semanal veio sua amizade com Lula.

%\textbf{Mino:} 
\normalfont
Ficamos logo amigos. Amigos de jantar um na casa do
outro.

%\textbf{Gianni:}
\itshape
 Você me levou algumas vezes para comer frango com
polenta na casa de Lula.

%\textbf{Mino:} 
\normalfont
Lembro. E a minha mulher, Angélica, gostava muito de
Marisa, mulher de Lula, e eu também simpatizava com ela. Era de
descendência italiana e Lula chamava a avó dela de \emph{nonnina}.
Marisa era uma mulher muito firme e vigorosa, uma rocha moral. Certa
vez, Lula me liga e diz: ``Vem jantar aqui, na Granja do Torto''. Aí
desenrolou"-se aquele enredo já contado. Combinamos uma grande entrevista
e fizemos de 13 páginas na \emph{CartaCapital}. Isso em 2005. Eu tinha,
um pouco antes, visitado o chefe da Polícia Federal, Paulo Lacerda,
delegado probo e elegante. Fui com Sergio Lirio e Belluzzo. E a certa
altura eu disse a Lacerda. ``A Operação Chacal pegou o disco rígido do
Daniel Dantas. E aí?'' A Operação Chacal visava Dantas envolvido nas
tramoias da privatização das comunicações, adversário da \versal{TIM} italiana,
dono do Banco Opportunity, com ramificações em paraísos fiscais. À minha
pergunta Lacerda respondeu que o disco rígido tinha sido entregue a
Ellen Gracie, ministra do \versal{STF}. Disse ainda: ``Está vendo aquela gaveta?
Uma cópia do disco rígido está ali dentro.'' E apontou para a mesa do
seu gabinete. Perguntei: ``O senhor recebeu pressões para enterrar a
Operação Chacal?'' Ah, muitas, de deputados, senadores.'' Ministros?
``Sim.'' Por exemplo, José Dirceu? ``Sim'', respondeu ele. Já contei
essa história. Perguntei ainda: ``Mas, desculpe, por que não abrem o
disco rígido?'' Ele disse: ``Se abrirem, acaba a República''.

%\textbf{Gianni:}
\itshape
 E como foi a entrevista de 13 páginas com Lula?

%\textbf{Mino:} 
\normalfont
Nessa entrevista, Lula me disse: ``Você sabe muito bem
que não sou de esquerda''. Eu pensei em Norberto Bobbio, na ideia de
quem quer igualdade é de esquerda. E indaguei: ``Não é esta a sua
preocupação? E Lula respondeu que, se assim fosse, ele era um convicto
esquerdista. O \versal{PT} nasceu como partido de esquerda, com ideário de
esquerda. Mas repito: o partido no poder portou"-se como os outros. Hoje,
o \versal{PT} está crescendo novamente, na esteira do Lula, é claro. Mas ainda é
um partido tíbio, sempre na defensiva.

%\textbf{Gianni:}
\itshape
 Já disse isso a Lula?

%\textbf{Mino:} 
\normalfont
Sempre. Além disso, ele me lê, sabe. Ele diz que os
trabalhadores brasileiros saberão reagir. Algum dia, talvez. Na hora H,
sempre estive ao lado dele. Em 1980, em companhia de Raymundo Faoro, fui
visitá"-lo no Dops. Fomos muito bem recebidos pelo então diretor da \versal{PF},
Romeu Tuma. Ele nos deixou à vontade no seu próprio gabinete, cuidando
de se retirar. Havia ali um canto com sofá e poltronas. Lá pelas tantas,
Faoro propôs: ``Olha, estou à disposição para ser seu advogado''. E Lula
disse: ``Não, doutor Faoro, isso é coisa pouco importante. Quem vai me
defender é o Greenwald, o senhor não se preocupe''. Depois, ele queria o
Faoro na chapa dele como candidato à Vice-Presidência, em 1998.

%\textbf{Gianni:}
\itshape
 Como estava Lula no Dops?

%\textbf{Mino:} 
\normalfont
Tranquilo. Mas estava sendo muito bem tratado. Romeu Tuma
não somente mandava chamar dona Marisa e os filhos todos os dias. Fui
com Angélica ao enterro da mãe de Lula, falecida durante a prisão do
filho. O preso veio, Tuma o fez acompanhar por dois agentes à paisana.
Todos os dias, no Dops, Lula comeu lulas fritas.

%\textbf{Gianni:}
\itshape
 Lula é uma força da natureza. Teve câncer na garganta.
Perdeu a mulher. E está sendo condenado sem provas.

%\textbf{Mino:} 
\normalfont
Sim, é um homem forte. Lidou com o câncer com bravura.
Perdeu a Marisa para a Lava Jato e está sendo julgado sem provas. Quando
o encontro, ele me diz que tenho de ficar sonhando, esperançoso, porque
a esperança não morre nunca. Estranhamente, um mês antes da morte de
Marisa, cheguei para Lula e disse: ``Sabe que eu tenho sonhado muito com
sua mulher. Me lembro do dia em que estávamos naquele bar, embaixo de
sua casa, quando você morava lá no alto, atrás da Volkswagen. Marisa
estava presente, e a certa altura chegou a perua da revista para me
pegar. Fui embora e você ficou no bar, tomando pinga com cambuci. Mas
Marisa saiu e ficou acenando. Eu sonho com essa cena: eu saio de carro e
ela saudando. É um sonho recorrente''. Ela morreu um mês depois.
